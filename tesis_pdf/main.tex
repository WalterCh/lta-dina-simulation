\documentclass[11pt,openany,a4paper]{book}
% ---------------------------------------------------------------------------- %

\usepackage[utf8]{inputenc}
\usepackage[spanish]{babel}

\usepackage{array}

\usepackage{amsfonts}
\usepackage{amsmath}
\usepackage{amssymb}
%\usepackage{bm}
\usepackage{background}
\usepackage{caption}
\captionsetup[table]{name=Tabla}
%\usepackage{enumerate}
\usepackage[T1]{fontenc}  % Para uso de tildes del teclado.
\usepackage{float}
\usepackage[colorlinks=true,urlcolor=red,citecolor=green,linkcolor=blue]{hyperref}
%\usepackage{indentfirst} 				% indentación del primer parrafo
\usepackage[top=2.54cm, bottom=2.54cm, left=3cm, right=2.54cm]{geometry} %margenes
\usepackage{graphicx}
\usepackage{lipsum}
\usepackage{listings}
\usepackage{longtable}
%\usepackage{lscape}
\usepackage{multicol}
\usepackage{multirow}
\usepackage[round]{natbib}              % libreria para la bibliografia
%\usepackage{pifont}
%\usepackage{ragged2e}
\usepackage{setspace}
\usepackage{subfigure} 					% uso de várias figuras en una sola
\usepackage{tabularx}
%\usepackage[small,compact]{titlesec} 	% encabezamiento de los títulos: menores y compactos
%\usepackage[nottoc]{tocbibind} 			% para que la bibliografia aparezca en el indice
\usepackage{url}
\usepackage{xcolor}
\definecolor{miverde}{rgb}{0,0.6,0}
\definecolor{migris}{rgb}{0.5,0.5,0.5}
\definecolor{mimalva}{rgb}{0.58,0,0.82}

% -----------------------------------librerias extras---------------------------------- %
\lstset{ %
	backgroundcolor=\color{white},   % Indica el color de fondo; necesita que se añada \usepackage{color} o \usepackage{xcolor}
	basicstyle=\footnotesize,        % Fija el tamaño del tipo de letra utilizado para el código
	breakatwhitespace=false,         % Activarlo para que los saltos automáticos solo se apliquen en los espacios en blanco
	breaklines=true,                 % Activa el salto de línea automático
	captionpos=b,                    % Establece la posición de la leyenda del cuadro de código
	commentstyle=\color{miverde},    % Estilo de los comentarios
	deletekeywords={...},            % Si se quiere eliminar palabras clave del lenguaje
	escapeinside={\%*}{*)},          % Si quieres incorporar LaTeX dentro del propio código
	extendedchars=true,              % Permite utilizar caracteres extendidos no-ASCII; solo funciona para codificaciones de 8-bits; para UTF-8 no funciona. En xelatex necesita estar a true para que funcione.
	frame=single,	                   % Añade un marco al código
	keepspaces=true,                 % Mantiene los espacios en el texto. Es útil para mantener la indentación del código(puede necesitar columns=flexible).
	keywordstyle=\color{blue},       % estilo de las palabras clave
	language=R,                 % El lenguaje del código
	otherkeywords={*,...},           % Si se quieren añadir otras palabras clave al lenguaje
	numbers=left,                    % Posición de los números de línea (none, left, right).
	numbersep=8pt,                   % Distancia de los números de línea al código
	numberstyle=\small\color{migris}, % Estilo para los números de línea
	rulecolor=\color{black},         % Si no se activa, el color del marco puede cambiar en los saltos de línea entre textos que sea de otro color, por ejemplo, los comentarios, que están en verde en este ejemplo
	showspaces=false,                % Si se activa, muestra los espacios con guiones bajos; sustituye a 'showstringspaces'
	showstringspaces=false,          % subraya solamente los espacios que estén en una cadena de esto
	showtabs=false,                  % muestra las tabulaciones que existan en cadenas de texto con guión bajo
	stepnumber=1,                    % Muestra solamente los números de línea que corresponden a cada salto. En este caso: 1,3,5,...
	stringstyle=\color{mimalva},     % Estilo de las cadenas de texto
	tabsize=1,	                   % Establece el salto de las tabulaciones a 2 espacios
	title=\lstname                   % muestra el nombre de los ficheros incluidos al utilizar \lstinputlisting; también se puede utilizar en el parámetro caption
}



\begin{document}
	%-------------------------------------------------------
	\frontmatter
	\onehalfspacing 
	
	%Caratula
	\begin{titlepage}
		\backgroundsetup{
			placement = center,
			angle = 0,
			scale = 1,
			contents = {\includegraphics{figuras/fondo_caratula.pdf}}
		}
		\begin{center}
			\vspace*{1cm}
			\textbf{{\Large PONTIFICIA UNIVERSIDAD CATÓLICA DEL PERÚ }}\\
			\vspace*{0.55cm}
			
			\textbf{{\Large ESCUELA DE GRADUADOS}}\\
			
			\vspace{1cm}
			\begin{figure}[H]
				\centering
				\includegraphics[scale=.6]{figuras/logo-pucp.pdf}
			\end{figure}
			\vspace{1cm}
			
			\begin{spacing}{1.5}
				\textsc{{\LARGE Modelos Longitudinales de Diagnóstico Cognitivo}}
			\end{spacing}
			\vspace{1.2cm}
			\textbf{{\large Tesis para optar el grado de Magister en Estadística}}\\
			
			\vspace*{1.2cm}
			\textbf{{\large Presentado por:}}\\
			\vspace*{0.3cm}
			\textbf{{\large Ronald Minchola }}\\
			\vspace*{1.2cm}
			\textbf{{\large Asesor:}}\\
			\vspace*{0.3cm}
			\textbf{{\large Dr. Luis Valdivieso}}\\
			
			\vspace*{1.2cm}
			
			\textbf{\large Miembros del jurado:}\\
			\vspace{5mm}
			\textbf{\large 
					Dr. Nombre completo jurado 1\\
					Dr. Nombre completo jurado 2\\
					Dr. Nombre completo jurado 3}
			
			\vfill
			
			\textbf{Lima, Diciembre 2019}
		\end{center}
	
\end{titlepage}
	
	%Dedicaoria
	\backgroundsetup{
	placement = center,
	angle = 0,
	scale = 1,
	contents = {\includegraphics{figuras/fondo_vacio.pdf}}
}

\chapter*{Dedicatoria}
	\lipsum[2] \lipsum[3] \lipsum[4] \lipsum[7]
	
	Con mucho cariño para mis padres:
	José Ronier Minchola Zavaleta e
	Hilda Mary Alza Silva 
	por haberme guiado desde pequeño en el 			mundo del estudio por haber dedicado mucho tiempo a mi formación académica y también como ser humano.
	Para Sinthia y mis pequeños hijos Sergio, Emilio y Valery que son fuente de inspiración siempre. Estoy seguro que algun dia leerán mi trabajo por su curiosidad hacia las ciencias.
	A mis hermanos Bruno y Aldo por su constante motivación para terminar con éxito esta gran proeza.
	
	
	
	%Agradecimientos
	\chapter*{Agradecimentos}
	\lipsum[2] \lipsum[3] \lipsum[4] \lipsum[7]
	A mis padres Hilda Mary y José Ronier con ese amor infinito que se tiene por los creadores, por haber inculcado en mi esas ganas de ser un profesional y mejor persona. A Sinthia por ser esposa y compañera por su comprensión y paciencia, a mis hijos Sergio, Emilio y Valery por ser fuente de inspiración siempre en cada paso que doy. Algún dia leeran este trabajo espero que si. 
Quiero agradecer a todos mis profesores desde el buhonero que me venia a enseñar fracciones, MCM y MCD cuando todavia era un niño alla en la lejana década de los ochentas en la desaparecida tienda de Comercial Rominza, pasando por todos mis profesores de las academias y del CEP Claretiano recordando al gran Osiel Linares con sus tizas de colores y sus Diagramas de Venn, hasta todos los profesores de la Maestria en Estadistica muy en especial al Dr. Luis Valdivieso Serrano por su calidad de persona y ser humano, además de su sabiduria y paciencia hacia mi persona
Agradecer también con mucho corazón a mis tios Efrain y mi dulce tia Josefina "Chepita" por haberme apoyado cobijandome del frio limeño mil gracias me han dado mucho cariño y enseñanzas diariamente.
    
	
	%Resumen
	\chapter*{Resumen}
	\lipsum[1] \lipsum[2] \lipsum[1] \lipsum[2] 
	
	\vspace{5mm}
	
	\noindent \textbf{Palabras clave:} palabra-clave1, palabra-clave2, palabra-clave3.
	
	%Abstract
	\chapter*{Abstract}
	.Abstract ..\lipsum[1] \lipsum[2] \lipsum[1] \lipsum[2] 
	
	\vspace{5mm}
	
	\noindent \textbf{Keywords:} keyword1, keyword2, keyword3.
	
	
	
	%Tabla de contenidos 
	\tableofcontents
	%----------------------------------------------------------
	
	%\chapter{Lista de Abreviaturas}
	\begin{tabular}{ll}
	 		fdp     & Función de densidad de probabilidad .\\
		pBF 		& Pseudo factor de Bayes(\emph{Pseudo bayes factor}).\\
	\end{tabular}
	
	\chapter{Lista de Símbolos}
	\begin{tabular}{ll}
			$\mu$    & Media.\\
	\end{tabular}
	
	% lista de Figuras
	%\listoffigures    
	
	% lista de cuadros
	%\listoftables                
	
	% ---------------------------------------------------------------------------- 
	
	
	
	%----------------------------------------------------------------------------
	\mainmatter
	
	%Introduccion
	% ------------------------------------------------------------------------- 
\chapter[Introducción]{Introducción} \label{cap:introduccion}
	
	\section{Consideraciones Preliminares} \label{sec:consideraciones}
	
		Los modelos de diagn\'{o}stico cognitivo (CDM) son modelos de clases latentes que se utilizan para clasificar a los encuestados en grupos homog\'{e}neos basados en m\'{u}ltiples variables latentes categ\'{o}ricas que representan a los atributos cognitivos medidos en una prueba. Uno de los modelos m\'{a}s populares de esta gran familia es el llamado modelo DINA, el cual tuvo su primera aparici\'{o}n con los trabajos de Haertel (1989) enfocados principalmente en el campo educacional. Este modelo considera respuestas observadas dicot\'{o}micas de parte de los examinados, variables predictoras latentes dicot\'{o}micas y tiene como restricci\'{o}n que los examinados deben de dominar obligatoriamente todas las habilidades requeridas para correctamente responder cada item, aquellas que se resumen en una matriz denominada Q. Asimismo, este modelo estima par\'{a}metros para los items, los cuales son denominados par\'{a}metros de ruido (estimaci\'{o}n de dos probabilidades de error): Adivinaci\'{o}n y Desliz, el primero es la probabilidad de responder correctamente a un item $j$ a pesar de no dominar las habilidades requeridas para hacerlo, mientras que el segundo es la probabilidad de fallar a un item a pesar de dominar las habilidades requeridas para hacerlo; adem\'{a}s el modelo DINA se encuentra clasificado como un modelo no compensatorio conjuntivo. Es no compensatorio porque requiere que cada una de las habilidades est\'{e} presente para producir una respuesta correcta al \'{i}tem (dicot\'{o}mico o polit\'{o}mico) y es conjuntivo cuando todas las habilidades requeridas para responder un \'{i}tem, necesariamente tienen que ser dominadas por el individuo para obtener la respuesta correcta.\\           
		
		En este proyecto se busca estudiar c\'{o}mo esta clasificaci\'{o}n pudiera verse afectada en el tiempo, cuando es factible el aplicar la prueba en repetidas ocasiones. 
		\setlength{\parskip}{6mm}
		
		Para ello, algunos autores como Kaya and Leite [2017] han propuesto un modelo que combina el modelo de an\'{a}lisis de transici\'{o}n de clases latentes (LTA) y los CDMs. De manera similar Li, F. et.al. [2016] han utilizado el LTA para proponer una metodolog\'{i}a de transici\'{o}n de clases con el modelo DINA, donde la estimaci\'{o}n de su modelo lo realizan usando m\'{e}todos bayesianos, a diferencia de los primeros autores que utilizaron el paquete Mplus implementado para analizar CDMs longitudinales junto con estudios de simulaci\'{o}n de Montecarlo. El objetivo de esta investigaci\'{o}n es analizar estas propuestas u otras relacionadas al problema e implementar la estimaci\'{o}n de sus par\'{a}metros. Se ilustrar\'{a} tambi\'{e}n todo ello con una aplicaci\'{o}n real y se espera que puedan realizarse, de ser posible, algunas variantes o extensiones del modelo para lidiar con las restricciones que tienen estos modelos, sobre todo en lo concerniente a la replicabilidad de la prueba original.
		
		Usualmente cuando queremos analizar datos longitudinales las preguntas de investigaci\'{o}n est\'{a}n en relaci\'{o}n al cambio a trav\'{e}s del tiempo, tal cambio, en nuestro modelo estar\'{a} referido a cambios en las clases cognitivas latentes de los individuos.
		
		El an\'{a}lisis de transici\'{o}n de clases latentes (LTA) es una aplicaci\'{o}n longitudinal del modelo de an\'{a}lisis de clases latentes, que tiene por objetivo identificar si existe un cambio entre las clases latentes a trav\'{e}s del tiempo (Collins y Lanza, 2010). Mientras que el an\'{a}lisis de clases latentes es aplicado a un conjunto de variables recolectadas en un tiempo espec\'{i}fico para identificar las clases latentes que presentan los patrones de respuesta en los datos, el an\'{a}lisis de transici\'{o}n de clases latentes es aplicado a mediciones repetidas en el tiempo para estudiar el movimiento entre las clases latentes identificadas (Collins y Lanza, 2010).
		
		El LTA fue desarrollado inicialmente para estudiar el cambio secuencial por etapas de un tipo de variables latentes llamadas variables latentes din\'{a}micas (Collins and Wugalter, 1992). Las variables latentes din\'{a}micas incluyen caracter\'{i}sticas tales como actitudes y patrones de personalidad que cambian a trav\'{e}s del tiempo. La t\'{e}cnica ha sido, posteriormente utilizada para el estudio sobre las intervenciones como por ejemplo el abandono del consumo de tabaco y la disminuci\'{o}n de malas conductas (comportamiento inapropiado). El an\'{a}lisis de transici\'{o}n latente es un m\'{e}todo muy usado para investigar el crecimiento acad\'{e}mico cuando las variables latentes son categ\'{o}ricas (Boscardin et al., 2008; Compton et al., 2008; Trentacosta et al., 2011) 
		
		En esta investigaci\'{o}n, describiremos el uso del LTA en combinaci\'{o}n con un modelo de diagn\'{o}stico cognitivo para analizar pruebas educativas largas que miden habilidades cognitivas m\'{u}ltiples. Esta nueva clase de modelos  que los llamamos de transici\'{o}n de clases latentes cognitivas (LTCA) permiten investigar hip\'{o}tesis acerca del efecto de una intervenci\'{o}n (por ejemplo, un programa instruccional o un cambio en la pol\'{i}tica educativa) mediante el cambio en las probabilidades de transici\'{o}n antes y despu\'{e}s de una intervenci\'{o}n con relaci\'{o}n al estado cognitivo.
		
		En este nuevo modelo se usan datos longitudinales para investigar si alg\'{u}n cambio ha ocurrido entre los estados latentes a trav\'{e}s del tiempo. Las transiciones son expresadas junto con sus probabilidades de cambio de un estado latente a otro. 
		
		El LTCA presenta varias razones para su uso: primero, que estos modelos pueden representar variables latentes multidimensionales, segundo pueden modelar y predecir el cambio a trav\'{e}s del tiempo, el cual es en cierto sentido discreto y tercero nos permiten conocer si algunos estados latentes pueden tener prevalencias muy bajas en etapas iniciales, pero a medida que los individuos hacen la transici\'{o}n en el tiempo, su prevalencia aumenta (Collins y Lanza, 2010). 
		
		Este nuevo modelo, entonces permitir\'{a} a un investigador responder directamente a un conjunto de preguntas tales como: ?`Existe alg\'{u}n cambio entre las clases latentes a trav\'{e}s del tiempo? y si la respuesta es positiva explicar ?`c\'{o}mo caracterizar dicho cambio a trav\'{e}s de las probabilidades?
		
		As\'{i}, como en el an\'{a}lisis de clases latentes, en este nuevo modelo se estiman tambi\'{e}n las probabilidades de respuesta al item, pero adem\'{a}s la prevalencia en las clases latentes que se entiende como el n\'{u}mero de casos de un evento en una poblaci\'{o}n en un momento dado y la incidencia de las transiciones entre las clases latentes junto con una medida del error.   
		
		Tres grupos de par\'{a}metros son estimados en un modelo LTCA. El primer grupo de par\'{a}metros est\'{a} compuesto por las probabilidades de transici\'{o}n entre las clases latentes. Estas son particularmente importantes porque ofrecen la soluci\'{o}n a la mayor\'{i}a de preguntas del tipo: ?`C\'{o}mo cambian los estados de dominio de las habilidades cognitivas latentes de etapa en etapa? El segundo grupo de par\'{a}metros estima el dominio o no de las digamos $C$ distintas habilidades cognitivas por parte de cada estudiante en su primera medici\'{o}n o etapa. El tercer grupo de par\'{a}metros estima la relaci\'{o}n entre el estado latente y la pregunta. Esto es, produce una probabilidad de respuesta correcta o incorrecta para cada item dado el dominio de los diferentes estados latentes.
		
		El segundo y el tercer grupo de par\'{a}metros proporcionan tambi\'{e}n la soluci\'{o}n a otra pregunta de investigaci\'{o}n: ?`Cu\'{a}l es el estado de dominio individual en cada habilidad cognitiva en cada etapa u ocasi\'{o}n? La respuesta a esta pregunta proporciona informaci\'{o}n \'{u}til para el estado del estudiante o paciente en cada punto en el tiempo.
		
		%% ------------------------------------------------------------------------- %%
		\section{Objetivos} \label{sec:objetivo}
		
		El objetivo general de la tesis es estudiar un nuevo modelo de transici\'{o}n de clases latentes de habilidades cognitivas para datos longitudinales, as\'{i} como observar su aplicaci\'{o}n a un conjunto de datos reales. De manera espec\'{i}fica:
		
		\begin{itemize}
			\item Revisar la literatura acerca del an\'{a}lisis de transici\'{o}n latente 
			\item Presentar los fundamentos y propiedades de los modelos de transici\'{o}n de clases latentes (LTA), los modelos de an\'{a}lisis de clases latentes (LCA) y los modelos de diagn\'{o}stico cognitivo (MDCs)
			\item Estudiar puntualmente el nuevo modelo de transici\'{o}n latente.
			\item Estudiar el proceso de estimaci\'{o}n de par\'{a}metros en los modelos de transici\'{o}n latentes bajo el enfoque cl\'{a}sico o frecuentista
			\item Realizar un estudio de simulaci\'{o}n a efectos de comparar la estimaci\'{o}n obtenida
			\item Aplicar el modelo a un conjunto de datos longitudinales  en el \'{a}rea educativa que involucren alg\'{u}n o algunos tipos de tareas cognitivas.
		\end{itemize}
		
		
		%% ------------------------------------------------------------------------- %%
		\section{Organizaci\'{o}n del Trabajo} \label{sec:organizacion}
		
		Para alcanzar los objetivos de la investigaci\'{o}n se considera la organizaci\'{o}n siguiente:
		
		En el Cap\'{i}tulo \ref{cap:introduccion} se presenta consideraciones generales acerca de los principales modelos de diagn\'{o}stico cognitivo entre los que destacan el modelo DINA, el modelo G-DINA y a su vez se estudia la literatura acerca del an\'{a}lisis de transici\'{o}n latente (LTA) y las cadenas de M\'{a}rkov. En el Cap\'{i}tulo \ref{cap:modelo} se estudia el modelo de transici\'{o}n de clases latentes cognitivas, la estimaci\'{o}n de sus par\'{a}metros y su implementaci\'{o}n computacional. En el capitulo \ref{cap:simulacion} se presentan los resultados de la aplicaci\'{o}n del modelo a un conjunto de datos longitudinales en el \'{a}rea educativa orientados a conocer alg\'{u}n o algunos tipos de tareas cognitivas. En el capitulo \ref{cap:aplicacion} se presentan algunas conclusiones, recomendaciones y sugerencias para futuras investigaciones que se podrian derivar de este trabajo. Se incluir\'{a} finalmente en anexos c\'{o}digo en R para la estimaci\'{o}n de los par\'{a}metros.
		
		
		
		
	%Conceptos
	%% ------------------------------------------------------------------------- 
\chapter[Modelos de clases latentes y diagn\'{o}stico cognitivo]{Modelos de clases latentes y diagn\'{o}stico cognitivo} \label{cap:concepto}
	En muchas \'{a}reas de estudio, como en las ciencias sociales, ciencias de la salud y particularmente en las ciencias del comportamiento como la psicolog\'{i}a, es de sumo inter\'{e}s aproximarnos a constructos te\'{o}ricos, los cuales si bien no pueden ser observados directamente, asumimos que tienen un efecto sobre variables que si pueden ser medidas, permiti\'{e}ndonos un acercamiento a estos constructos y un mejor entendimiento acerca de las variables medidas.
	Estos constructos son com\'{u}nmente llamados variables latentes en la literatura estad\'{i}stica y su modelo estad\'{i}stico especifica la distribuci\'{o}n conjunta de las variables observadas o llamadas tambi\'{e}n manifiestas (indicadores) y las variables latentes.
	Por otro lado, los modelos de diagn\'{o}stico cognitivo son modelos psicom\'{e}tricos usados para evaluaci\'{o}n de estudiantes en cuanto a sus perfiles o clases y que nos permiten una medici\'{o}n efectiva del aprendizaje as\'{i} como tambi\'{e}n una mejor instrucci\'{o}n y posiblemente pol\'{i}ticas de intervenci\'{o}n educativa para hacer frente a las necesidades individuales y grupales de los estudiantes.
	Es por eso, que se pasar\'{a} a estudiar todo lo concerniente a la integraci\'{o}n de estos conceptos         
	
	\section{Modelos de clases latentes}
		El modelo de clases latentes es un caso particular de los modelos de variables latentes en el cual tanto las variables observables o manifiestas como las no observables o latentes son categ\'{o}ricas. La idea es poder agrupar a los individuos en clases seg\'{u}n las variables observables. Se considera que la variable latente es categ\'{o}rica como producto de una evidencia a priori o te\'{o}rica, o simplemente tomada as\'{i} por una cuesti\'{o}n pr\'{a}ctica. Para mayor detalle ver Lazarsfeld and Henry (1968).
		Las ventajas del an\'{a}lisis de clases latentes son varios: reducen la complejidad de un conjunto de datos agrup\'{a}ndolos en clases, permite estimar varias probabilidades, y permite analizar datos categ\'{o}ricos sin la necesidad de aplicar transformaciones.
		Se comienza a partir de dos supuestos para la versi\'{o}n est\'{a}ndar del modelo. Dentro de cada clase, todos los individuos tienen las mismas probabilidades de respuesta a las variables manifiestas.
		Se cumple la independencia condicional; es decir, las respuestas entre los individuos son independientes dado que estos pertenecen a una misma clase.
		Para tener una idea analicemos primero un modelo de clases latentes con variables manifiestas binarias, en el que la variable latente $L$ es unidimensional y presenta $C$ categor\'{i}as o clases que los denotaremos con el \'{i}ndice $c=1,2,\ldots C$. Sea ${ \pi  }_{ ic }$ la probabilidad de obtener una respuesta positiva a la variable manifiesta $ X_{ i }$, dado que el individuo $i$ est\'{e} en la clase $c$; es decir, $\pi_{i c}=P\left(X_{i}=1 | L=c\right)$ y sea ${ \eta  }_{ c }$ la probabilidad a priori de que un individuo pertenezca a la clase $c$. Entonces la funci\'{o}n de probabilidad conjunta del vector de respuestas manifiestas o patrones de respuesta $\boldsymbol{X}=( X_{ 1 }, X_{ 2 },\ldots , X_p)$ viene dada por:
		\begin{equation}
			f(\mathbf{x}) = P(\mathbf{X}=\mathbf{x}) = \sum_{c=1}^{C} P(\mathbf{X}=\mathbf{x} | L=c ) P(Y=c) = \sum_{c=1}^{C} g(\mathbf{x} | c) \eta_{c},
			\label{2.1}
		\end{equation}
		donde
		\begin{equation}
			g(\mathbf{x} | c) = \prod_{i=1}^{p} \pi_{ic}^{x_{i}}\left(1-\pi_{i c}\right)^{1-x_{i}}
			\label{2.2}
		\end{equation}
		y $x_{i}$ es la respuesta 1 o 0 a $X_{i}$.\\
		
		De esta manera, cada individuo es descrito por dos dimensiones: el vector de variables manifiestas $\boldsymbol{X}$ y el indicador de pertenencia a la clase.
		La probabilidad a posteriori de que un individuo con un patr\'{o}n de respuestas $\boldsymbol{x}=(x_{1}, x_{2},\ldots, x_{p})'$ pertenezca a la clase 
		$j$ es por tanto:
		
		\begin{equation}
			h(c | \mathbf{x}) = P(L=c | \mathbf{X}=\mathbf{x}) = \frac{\eta_{c} \prod\limits_{i=1}^{p} \pi_{i c}^{x_{i}}\left(1-\pi_{i c}\right)^{1-x_{i}}}{\sum_{c=1}^{K} \eta_{c} \prod_{i=1}^{p} \pi_{i c}^{x_{i}}\left(1-\pi_{i c}\right)^{1-x_{i}}} .
			\label{2.3}
		\end{equation}
		
		Esta funci\'{o}n es utilizada luego como regla de clasificaci\'{o}n para asignar a un individuo a la clase que tiene mayor probabilidad de pertenencia, dado el patr\'{o}n de respuestas que tenga.
		El modelo requiere de la estimaci\'{o}n de los par\'{a}metros ${ \eta  }_{ c }$ y ${ \pi  }_{ ic }$, y el c\'{a}lculo del ajuste del modelo. Otro aspecto es tambi\'{e}n identificar las clases latentes subyacentes e interpretarlas de manera que tengan un sentido con los datos.
		
	\section{Cadenas de Markov}
		Extenderemos ahora el modelo de clases latentes a datos de caracter longitudinal para averiguar entre otras cosas como cambian, seg\'{u}n los datos, la pertenencia de un individuo a una clase. Para ello requeriremos del concepto de cadenas de Markov que pasamos brevemente a revisar.\\ 
		
		Consideremos un conjunto $E$ finito o numerable. Sea $X_{0}, X_{1}, X_{2},\ldots $ una sucesi\'{o}n de variables aleatorias discretas las cuales toman valores en $E$ y se asumen est\'{a}n definidas en un espacio probabilistico $(\Omega,A,P)$. El conjunto $E$ se denominar\'{a} espacio de estados, y sus elementos estados.
		Las variables aleatorias ${X}_{0}, X_{ 1 },\ldots , X_{ n }$ se dicen independientes si y s\'{o}lo si, se cumple que:
		
		\begin{equation}
		P( X_{ 0 }= X_{ { 0} }, X_{ 1 }= X_{ { 1 }},\ldots , X_{ n }= X_{ { n } })=\prod _{ k=1 }^{ n }{ P( X_{ k }= X_{ k }) } ,       \label{2.4}
		\end{equation}
		
		para todo estado  ${x}_{0}, X_{ 1 },\ldots , X_{ n }$. En este caso los eventos $A=\left\{  X_{ 0 }= X_{ 0 },\ldots , X_{ n }= X_{ n } \right\} $ y $B=\left\{  X_{ n+1 }= X_{ n+1 },\ldots , X_{ n+m }= X_{ n+m } \right\}$ son independientes para cualesquier $n=0,1,\ldots $ y $m=1,2,\ldots $ y cualquiera sea la sucesi\'{o}n de estados $ X_{ 0 },\ldots , X_{ n+m }$. La dependencia markoviana consiste en que la probabilidad del suceso B depende s\'{o}lo del valor que toma la variable aleatoria $ X_{ n }$ y no de los valores que toman las variables aleatorias $X_{ 0 },\ldots , X_{ n-1 }$. Si el \'{i}ndice de la sucesi\'{o}n representa el tiempo y $n$ es el instante presente en una cadena de Markov, podemos decir que la probabilidad de que un suceso ocurra en los instantes futuros $n+1$,\ldots ,$n+m$, depende solamente del estado en que se encuentra la sucesi\'{o}n en el instante presente $n$ y no de los estados en que se encontr\'{o} en los instantes pasados $0,1,\ldots ,n-1$.\\    
		Formalmente diremos que la sucesi\'{o}n $\left\{  X_{ n } \right\}$ es una cadena de Markov con espacio de estados discreto si:
		
		\begin{align}
			P( X_{ n+1 } & = X_{ n+1 }| X_{ n }= X_{ n },\ldots , X_{ 0 }= X_{ 0 })= \ldots  \nonumber \\
			&=P( X_{ n+1 }= X_{ n+1 }| X_{ n }= X_{ n }), \label{2.5}
		\end{align}
		
		para todo $n = 1,2,\ldots $ y cualquier sucesi\'{o}n de estados $ X_{ 0 },\ldots , X_{ n+1 }$ en $E$, siempre que $P( X_{ n }= X_{ n },\ldots , X_{ 0 }= X_{ 0 })>0$. A la identidad (2.5) se le llama la propiedad de M\'{a}rkov. En adelante para no recargar notaciones asumiremos, sin p\'{e}rdida de generalidad, que el espacio de estados $E$ es un subconjunto de $\aleph $.
		En tal sentido, diremos que una cadena de Markov es homog\'{e}nea en el tiempo si para todo par de estados $i,j$, la probabilidad condicionada $P( X_{ n+1 }=j| X_{ n }=i)$ no depende de $n$, es decir,
		
		\begin{equation}
			P( X_{ 1 }=j| X_{ 0 }=i) = P( X_{ 2 }=j| X_{ 1 }=i)=\ldots = P( X_{ n+1 }=j| X_{ n }=i). \label{2.6} 
		\end{equation}
		
		En general, utilizamos la expresi\'{o}n cadena de M\'{a}rkov para referirnos a una cadena de Markov homog\'{e}nea en el tiempo.
		La caracterizaci\'{o}n de la cadena en este caso se definir\'{a} a trav\'{e}s de las probabilidades:
		
		\begin{equation}
			p_{ij} = P( X_{ 1 }=j| X_{ 0 }=i) \label{2.7}
		\end{equation} 
		
		y
		 
		\begin{equation*}
			\pi_{i}^{0}=P( X_{0}=i) 
		\end{equation*}  
		
		A la matriz $ P = [p _{ij}], i\in E,j\in  E$ (posiblemente infinita) se le denomina la matriz de transici\'{o}n, y al vector $\pi^{0} =({ \pi  }_{ i }^{ 0 })$, la distribuci\'{o}n inicial de la cadena de Markov. Es sencillo, ver que la matriz de transici\'{o}n satisface las siguientes propiedades:
		
		\begin{enumerate}
			\item[i) ] $p_{ ij }\ge 0$ para todo par de estados $i,j$ en $E$
			\item[ii) ] $\sum _{ j }^{  }{ p_{ ij }=1 }$ para todo estado $i$ en $E$
		\end{enumerate}		
	
		\noindent
		Por su parte, la distribuci\'{o}n inicial satisface las propiedades:
		
		\begin{enumerate}
			\item[i) ] ${ \pi  }_{ i }^{0}\ge 0$ para todo estado $i$ en E,
			\item[ii) ]  $\sum_{ i }{ { \pi  }_{ i }^{0} }=1$
		\end{enumerate}
		
		La definici\'{o}n de cadena de Markov incluye como caso particular, las sucesiones de variables aleatorias independientes e id\'{e}nticamente distribuidas y tambi\'{e}n las sucesiones formadas por sumas parciales de variables aleatorias independientes e id\'{e}nticamente distribuidas, cuando dichas variables aleatorias toman valores enteros.
		Consideremos ahora las probabilidades de transici\'{o}n de orden n de una cadena Markov $\left\{  X_{ k } \right\} $ con espacios de estados $E$, matriz de transici\'{o}n $P$ y distribuci\'{o}n inicial $\pi^{0}$. Dados dos estados $i,j$ sean:
		
		\begin{equation}
		p_{ ij }^{ (n) }=P( X_{ n }=j| X_{ 0 }=i) \quad \mbox{ y } \quad \quad { \pi  }_{ i }^{( n) }=P( X_{ n }=i), \label{2.8}
		\end{equation}
		
		a la matriz $p^{( n) }=[p_{ ij }^{ (n) }],$ se le llama la matriz de transici\'{o}n de orden n y al vector ${ \pi  }^{ n }=({\pi}_{ i }^{ n })$ la distribuci\'{o}n de probabilidad en el periodo $n$ de la cadena de Markov. Observemos que ${ \pi  }^{ 0 }$ es la distribuci\'{o}n inicial de la cadena de Markov y $P=p^{( 1) }$ es su matriz de transici\'{o}n. 
		Las probabilidades de transici\'{o}n de orden $n$ satisfacen la ecuaci\'{o}n de Chapman-Kolgomorov:
		
		\begin{equation}
			p_{ ij }^{ (m+n) }=\sum _{ k }^{  }{ p_{ ik }^{ (m) }p_{ kj }^{ (n) } } ,     \label{2.9}
		\end{equation}
		
		para todo par de indices $m,n$ y todo par de estados $i,j$. Esto es, en notaci\'{o}n matricial, (\ref{2.9}) tiene la forma
		
		\begin{equation}
			p^{ (m+n) }=p^{ (m) }\times p^{(n) }, \label{2.10}
		\end{equation}
		
		donde $x$ denota el producto de matrices. En efecto, aplicando la f\'{o}rmula de la probabilidad total y la propiedad de Markov (\ref{2.5}) se obtiene: 
		
		\begin{align}
			p_{ij}^{(m+n)} &= P( X_{ m+n }=j| X_{ 0 }=i)=\sum _{ k\ }^{  }{ P( X_{ m+n } } =j, X_{ m }=k| X_{ 0 }=i)= \nonumber \\
			&=\sum _{ k\ }^{  }{ P( X_{ m+n } } =j| X_{ m }=k)P( X_{ m }=k| X_{ 0 }=i)=\sum _{ k\epsilon E  }^{  }{ p_{ ik }^{ (m) } } p_{ kj }^{ (n) }
		\end{align}
		
		An\'{a}logamente se prueba que para la distribuci\'{o}n de probabilidad en el instante $n$ tiene lugar la identidad
		
		\begin{equation}
			{ \pi  }_{ j }^{( m+n) }=\sum _{ k }^{  }{ { \pi  }_{ k }^{( m) } } p_{ kj }^{( n) } , \label{2.11}
		\end{equation}
		
		para todo par de indices $m,n$ y todo estado $i$. En notaci\'{o}n matricial esto se escribe como:
		
		\begin{equation}
			{ \pi  }^{ (m+n) }={ \pi  }^{ (m) }\times p^{(n) }\nonumber
		\end{equation} 
		
		En particular, de (\ref{2.10}) resulta que $p^{(n) }=P\times p^{ (n-1) }$ y  $p^{(n)}=P^{n}$. En conclusi\'{o}n, la matriz de transici\'{o}n de orden $n$ es la potencia n-\'{e}sima de la matriz de transici\'{o}n $P$, y es correcto interpretar el super\'{i}ndice $n$ en la notaci\'{o}n $p^{ n }$ como la potencia n-\'{e}sima de la matriz $P$. La distribuci\'{o}n de probabilidad en el instante $n$ se obtiene mediante la f\'{o}rmula
		
		\begin{equation}
			\pi^{n}=\pi^{0} \times p^{ n } ,   \label{2.12}
		\end{equation}
		
		que tambi\'{e}n se escribe como:
		
		\begin{equation}
			\pi_{ j }^{ n }=\sum _{ k }^{  }{\pi^{0}_{ k }p_{ kj }^{ n-1 } } , \label{2.13}
		\end{equation}
		
		Calculemos ahora, para una elecci\'{o}n de \'{i}ndices ${ n }_{ 1 },\ldots ,{ n }_{ k }$ arbitraria, la distribuci\'{o}n del vector aleatorio $( X_{ { n }_{ 1 } },\ldots  X_{ { n }_{ k } })$, que llamamos la distribuci\'{o}n finito-dimensional de la cadena de Markov. Es claro que para este c\'{a}lculo es suficiente conocer las probabilidades de la forma 
		
		\begin{equation}
			P( X_{ 0 }\epsilon { A }_{ 0 },\ldots , X_{ n }\epsilon { A }_{ n })\quad \quad (n=0,1,\ldots ), \nonumber
		\end{equation}
		
		donde ${ A }_{ 0 },\ldots ,{ A }_{ n }$ son subconjuntos arbitrarios de E. A su vez, para calcular estas \'{u}ltimas probabilidades, es suficiente, dada una sucesi\'{o}n de estados ${ i }_{ 0 },\ldots ,{ i }_{ n }$, conocer las probabilidades
		
		\begin{align}
			P( X_{ n } ={ i }_{ n },\ldots , X_{ 0 }={ i }_{ 0 }) & = \nonumber \\
			& =P( X_{ n }={ i }_{ n }| X_{ n-1 }={ i }_{ n-1 })\ldots P( X_{ 1 }={ i }_{ 1 }| X_{ 0 }={ i }_{ 0 })P( X_{ 0 }={ i }_{ 0 }) \nonumber \\ 
			&=p_{ { i }_{ n-1 }{ i }_{ n } }\ldots p_{ { i }_{ 0 }{ i }_{ 1 } }{ \pi  }_{ { i }_{ 0 } } .\label{2.14} 
		\end{align}
		
		Este \'{u}ltimo c\'{a}lculo muestra que las distribuciones finito dimensionales de una cadena de Markov pueden determinarse si se conocen la matriz de transici\'{o}n $P$ y su distribuci\'{o}n inicial $\pi^{0}$. M\'{a}s a\'{u}n, es posible demostrar que la probabilidad de un suceso que depende de una cantidad arbitraria, no necesariamente finita, de variables aleatorias de la cadena de Markov, se puede hallar a partir de $P$ y de $\pi^{0}$ y que reciprocamente, dados un conjunto $E$, finito o numerable, una matriz $P=[ p _{ ij }], i\epsilon E,j\epsilon E$ que cumple las propiedades i) y ii) y un vector $\pi =({ \pi  }_{ i })_{i\epsilon E}$ que satisface las propiedades i) y ii) existe un espacio de probabilidades y una cadena de Markov homog\'{e}nea con espacio con espacio de estados $E$ y variables aleatorias definidas en este espacio de probabilidades, que tiene a $P$ como matriz de transici\'{o}n y a $\pi^{0}$ como distribuci\'{o}n inicial. De aqu\'{i} se obtiene la siguiente conclusi\'{o}n: Las propiedades probabil\'{i}sticas de una cadena de Markov, dependen \'{u}nicamente de la matriz transici\'{o}n y de la distribuci\'{o}n inicial de la cadena de Markov.
		Cuando es necesario escribir de forma expl\'{i}cita la distribuci\'{o}n inicial $\pi^{0}$ de una cadena de Markov, escribimos $p_{ { \pi  }^{ 0 } }$ en lugar de $P$. En particular, si esta distribuci\'{o}n inicial est\'{a} concentrada en un estado $i$ escribimos $p_{ i }$ en vez de $p_{ { \pi  }^{ 0 } }$ y diremos que la cadena de Markov parte de $i$ ya que $p_{ 0 }( X_{ 0 }=i)=1$.
		Consideremos un suceso arbitario de la forma

		\begin{equation}
			A=\left\{  X_{ { n }_{ 1 } }={ i }_{ { n }_{ 1 } },\ldots , X_{ { n }_{ k } }={ i }_{ { n }_{ k } } \right\} \nonumber
		\end{equation}
		
		Observamos que las probabilidades $p_{ { \pi  }^{ 0 } }$ y $p_{ i }$ correspondientes a la cadena de Markov con la misma matriz de transici\'{o}n, satisfacen la igualdad:
		
		\begin{equation}
			p_{ \pi^{0} }(A| X_{ 0 }=i)=p_{ i }(A), \nonumber
		\end{equation}
		
		\noindent
		como resulta de hacer ${ \pi^{0} }_{ { i }_{ 0 } }=1$ en la f\'{o}rmula (\ref{2.5}). Adem\'{a}s, tambi\'{e}n se cumple:
		
		\begin{equation}
			p_{ \pi  }(A)=\sum _{ i\epsilon E}^{  }{ p_{ \pi^{0} } } (A| X_{ 0 }=i){ \pi  }_{ i }=\sum _{ i\epsilon E}^{  }{ p_{ i }(A){ \pi  }_{ i } } ,  \label{2.15}
		\end{equation}
		
		\noindent
		como resulta de aplicar el teorema de probabilidad total.
		
	\section{Modelos de transici\'{o}n de clases latentes}
	\label{sec:An\'{a}lisis de transici\'{o}n latente }
	
		\subsection{Introducci\'{o}n}
			El modelo de transici\'{o}n de clases latentes llamado tambi\'{e}n de mixtura de clases latentes de Markov es un modelo de variable latente similar al de clases latentes (LCA). Sin embargo, mientras que el LCA usa datos de corte transversal, el LTA es usado con data tipo panel o longitudinal para investigar si alg\'{u}n cambio ha ocurrido dentro de las clases latentes a trav\'{e}s del tiempo.
			Cuando se dispone de datos longitudinales el modelo de transici\'{o}n de clases latentes permite direccionar un conjunto de preguntas: ?`Existe alg\'{u}n cambio entre las clases latentes a trav\'{e}s del tiempo? Si es asi como se puede caracterizar este cambio? Si un individuo se encuentra en una clase latente particular en el tiempo $t$, cu\'{a}l es la probabilidad de que este individuo permanezca en la misma clase latente en el tiempo $t+1$, y cual la probabilidad de que el individuo este en una clase latente diferente. El LTA es una forma de ajustar modelos que abordan este conjunto de preguntas adicionalmente al conjunto de preguntas abordadas. 
			Es por eso que esta investigaci\'{o}n propone, no solo estudiar las clases latentes de algunos fen\'{o}menos educativos, sino tambi\'{e}n observar como los individuos transitan por las diferentes clases latentes a trav\'{e}s del tiempo. El LTA al igual que el LCA estima probabilidades de respuesta al item. Por lo tanto, se estiman tambi\'{e}n la prevalencia de las clases latentes y la incidencia de las transiciones entre las clases latentes mientras se ajusta un error de medici\'{o}n.     
			El modelo de transici\'{o}n de clase latente con solo dos puntos en el tiempo se plantea como:     
			
			\begin{equation}
				P(Y=y)=\sum _{ {c}_{1}=1 }^{C}{\sum _{ {c}_{2}=1 }^{C}{\delta _{c_1} \tau_{c_2|c_1} }}\prod _{ t=1 }^{ 2 }{  \prod_{ j=1 }^{J}{\prod_{ r_{j,t}=1 }^{R_j}{\rho_{j,r_{j,t}|c_t}^{I(y_{j,t}=r_{j,t})}  } }    }, \label{2.16}
			\end{equation}
			
			\noindent
			donde
			
			\begin{equation}
				\sum _{ c_1 =1 }^{C}{\delta _{c_1} } =1,
				\sum _{ { c }_{ 2 }=1 }^{ C }{ { \tau  }_{ c2|c1 } } =1,
				\sum _{ r_{j,t}=1 }^{R_j}{\rho_{j,r_{j,t}|c_t} }  =1 ,\label{2.17}
			\end{equation}
			
			siendo $Y = [Y_ {j,t}]$ la matriz de respuestas de cualquier individuo, $Y_{j,t}$, su respuesta al item $j$ en el tiempo $t$, $C$ el n\'{u}mero de clases latentes, $J$ el n\'{u}mero de items, $\delta _{c1}$ la probabilidad de pertenencia del individuo a la clase latente $c_1$ en el tiempo $1$, $\tau_{c2|c1}$ la probabilidad de transici\'{o}n de que el individuo pase de la clase latente $c_1$ en el tiempo $1$ a una clase $c_2$ en el tiempo $2$, $r_{j, t}$ la categor\'{i}a de respuesta al item $j$ en el tiempo $t$, $I(y_{j,t}=r_{j,t})$ una variable indicadora que es igual a $1$ si la respuesta al item $j$ en el tiempo $t$ es  $r_{j,t}$ y $0$ en caso contrario, y $\rho_{j,r_{j,t}|c_t}$, es la probabilidad de que su respuesta al item $j$ en el tiempo $t$ sea $r_{j,t}$ condicionado a la membresia a la clase latente $c_t$. Se asume que cada item $j$ tiene ${ R }_{ j }$ posibles categor\'{i}as de respuestas. La consideraci\'{o}n de un modelo con m\'{a}s de dos puntos en el tiempo es inmediata pero esta implica el incremento en el n\'{u}mero de par\'{a}metros, no s\'{o}lo en t\'{e}rminos de las probabilidades de transici\'{o}n sino tambi\'{e}n de los par\'{a}metros dentro de las probabilidades $\rho_{j,r_{j,t}|c_t}$ de respuesta al item del modelo.
			
			\noindent
			\textbf{Ejemplo}\\
			Supongamos como ilustraci\'{o}n que se tienen $J=6$ variables observables binarias, que han sido medidas en los tiempos $t=1,2$. Por ejemplo, si estamos estudiando la delincuencia en adolescentes, las variables observadas podrian ser 6 items del cuestionario de delincuencia, donde cada variable observada $j$ asumimos tiene ${r}_{j,t }=1,2$ categor\'{i}as de respuesta (si o no) en cada tiempo. La tabla de contingencia cruzando las 6 variables con los dos tiempos tiene ${ 2 }^{ 2\times 6 }=4,096$ celdas o patrones de respuesta. Los patrones de respuesta en estas, son representados como $({ r }_{ 1,1 },\ldots ,{r}_{6,2})$. Por ejemplo, un patr\'{o}n de respuesta podria ser (si,no,no,no,no,no,si,si,no,no,no,no) que representa la respuesta de $"si"$ para el primer item del cuestionario en el tiempo 1 hasta una respuesta de no para el sexto item en el tiempo 2.\\
			Sea $L$ como antes, la variable latente categ\'{o}rica subyacente que tiene $C$ estados latentes. Sea ${L}_{1}$ que representa la variable latente categ\'{o}rica en el tiempo 1, donde ${ c }_{ 1 }=1,\ldots ,C$, ${L}_{2}$ representa la variable latente categ\'{o}rica en el tiempo 2, donde ${ c }_{ 2 }=1,\ldots ,C$ en el tiempo 2 y as\'{i} sucesivamente hasta ${ L }_{ T }$ que representa a la variable latente categ\'{o}rica en el tiempo $T$, con ${ c }_{ T }=1,\ldots ,C$. 
			
			Las prevalencias en las clases latentes son estimadas para cada punto en el tiempo. As\'{i}, por ejemplo para un modelo con dos puntos en el tiempo como el anterior y con cuatro clases latentes, ocho prevalencias requerir\'{a}n ser estimadas en total, mostrando ellas la proporci\'{o}n de personas en cada uno de los estados latentes. Por lo tanto, observando la prevalencia en estos estados podemos ver el aumento o disminuci\'{o}n en las prevalencias entre los puntos en el tiempo considerados en el estudio, asimismo observar patrones de cambio y como las personas se mueven entre los estados.
			
			Las probabilidades de respuesta al item representan las probabilidades de que las personas den respuestas correctas a cada item en particular sujeto a su clase de pertenencia. El modelo de transici\'{o}n de clase latente es particularmente usado m\'{a}s como un m\'{e}todo exploratorio, en donde el etiquetado o codificaci\'{o}n  de los estados latentes se realiza evaluando las probabilidades de respuesta al item. Existe una probabilidad de respuesta al item para cada combinaci\'{o}n item-estado. Por lo tanto, el n\'{u}mero de celdas en una tabla de probabilidades de respuesta al item es el n\'{u}mero de items multiplicado por el n\'{u}mero de estados latentes.
			
			Las probabilidades de transici\'{o}n son de relevante importancia porque permiten identificar si ocurri\'{o} alg\'{u}n cambio entre los estados latentes a trav\'{e}s del tiempo. Para un modelo con dos puntos en el tiempo $(T=2)$, las probabilidades de transici\'{o}n permiten la creaci\'{o}n de una tabla de clasificaci\'{o}n cruzada con el n\'{u}mero de estados latentes en el tiempo o etapa 1 y el n\'{u}mero de estados latentes en el tiempo o etapa 2. En general, dado cierto n\'{u}mero de puntos en el tiempo $T$, el modelo de transici\'{o}n de clase latente estima $T-1$ matrices de probabilidades de transici\'{o}n. Cada examinado, se asume que puede ser miembro de uno y solo un estado latente en cada punto en el tiempo.
			
			El modelo de transici\'{o}n de clases latentes asume la no existencia de datos faltantes en las variables indicadoras observadas. Sin embargo, los datos faltantes en el modelo LTA es sobrellevada de la misma forma que en el modelo de clases latentes LCA. Los datos pueden usarse a partir de las respuestas que dan los individuos a algunas de las preguntas en un tiempo en particular y para aquellos que solo est\'{a}n presentes en un subconjunto en el tiempo de medici\'{o}n. Se hace entonces la cl\'{a}sica suposici\'{o}n datos faltantes (missing at random).
			
			En el presente estudio, a las clases latentes del modelo de transici\'{o}n las llamaremos estados latentes teniendo en cuenta que las clases latentes son estados temporales, y que los individuos pueden moverse hac\'{i}a dentro y fuera de estos estados.
			
			La combinaci\'{o}n de grandes grados de libertad y la extrema dispersi\'{o}n no permiten probar hip\'{o}tesis tradicionales acerca del ajuste absoluto del modelo de transici\'{o}n de clases latentes. Esto se debe a que la distribuci\'{o}n de la estad\'{i}stica ${ G }^{ 2 }$(radio de verosimilitud) no es muy bien aproximada por una distribuci\'{o}n Ji-cuadrado y por consiguiente los valores $p$, son bastante inexactos. Es por eso que se prefiere enmarcar la selecci\'{o}n del modelo en t\'{e}rminos relativos donde sea posible, es decir, ajustar una serie de modelos y confiar en los AIC y BIC para tomar decisiones acerca del modelo que represente mejor a nuestros datos. Las pruebas de hip\'{o}tesis son usadas cuando se desea comparar modelos de transici\'{o}n anidados. Tambi\'{e}n la parsimonia y la conceptualizaci\'{o}n te\'{o}rica son importantes criterios de selecci\'{o}n en un modelo LTA, tal como lo es en un modelo de clase latente.
			
			Como dijimos anteriormente en el modelo LTA interesan estimar 3 conjuntos diferentes de par\'{a}metro.
		
		\subsection{Las prevalencias de los estados latentes}
			La prevalencia del estado latente $c$ en el tiempo $t$ se denota por ${\delta  }_{ ct }$ , y viene dada por la probabilidad de membres\'{i}a al estado latente $c$ en el tiempo $t$. En el ejemplo anterior hay 2 estados latentes en cada uno de los dos tiempos, y por tanto 4 prevalencias. Se tendr\'{a} una probabilidad de la membres\'{i}a al estado latente 1 en el tiempo 1.
			Debido a que los estados latentes son mutuamente excluyentes y exhaustivos en cada periodo de tiempo, es decir, que cada individuo es miembro de uno y solo un estado latente en el tiempo $t$, se debe cumplir que:
			
			\begin{equation}
				\sum_{ c_t =1 }^{C}{\delta _{c_t} } =1 .
				\label{2.18}
			\end{equation}
			
			En otras palabras, dentro de un tiempo en particular $t$, la prevalencia en los estados latentes debe sumar 1.
			
			\subsection{Las probabilidades de respuesta al item}
			La probabilidad de obtener una respuesta ${ r }_{ j,t }$ a la variable observada $j$, condicionada a la membres\'{i}a en el estado latente ${ c }_{ t }$ en el tiempo $t$ se denota mediante ${ \rho  }_{ j,{ r }_{ j,t }|{ c }_{ t } }$. Para cada combinaci\'{o}n del estado latente $c_{t}$ de la variable observada $j$ en el tiempo $t$, hay $R_{j}$ probabilidades de respuesta al item. Nuevamente cada individuo proporciona una y sola una respuesta alternativa a la variable $j$ en un tiempo espec\'{i}fico $t$ y por tanto se debe cumplir que: 
			
			\begin{equation}
				\sum _{ { r }_{ j,t }=1 }^{ { R }_{ j } }{ { \rho  }_{ j,{ r }_{ j,t }|{ s }_{ t } }=1 }, \label{2.19}
			\end{equation}
			
			para todo $j,t$. En otras palabras, para los individuos en el estado latente ${ c }_{ t }$ en el tiempo $t$, las probabilidades de cada respuesta alternativa a la variable $j$ deben sumar 1.
			
		\subsection{Las probabilidades de transici\'{o}n}
			La probabilidad de transici\'{o}n a un estado latente $i$ en el tiempo $T$, condicionada a su membresia al estado latente $j$ en el tiempo $T-1$ se denota mediante ${ \tau  }_{ i|j }$.
			Las $\tau$ s por lo general son acomodadas en una matriz de probabilidades de transici\'{o}n de la siguiente manera:
			
			\begin{equation} 
				\left[ \begin{array}{cccc} 
					\tau_{1|1} & \tau_{2|1} & \ldots & \tau_{c|1} \\  
					\tau_{1|2} & \tau_{2|2} & \ldots & \tau_{c|2} \\    
					\vdots & \vdots & \ddots & \vdots \\
					\tau_{1|c} & \tau_{2|c} & \ldots & \tau_{c|c} \\    
				\end{array} \right] , \label{2.20}
			\end{equation}
			
			donde ${ \tau  }_{ i|j }=P({ L }_{ T }=i|{ L }_{ T-1 }=j)$.\\
			
			Los individuos que se encuentran en un estado latente ${c }_{ T }$ en el tiempo $T$, estar\'{a}n en solo un estado latente en el tiempo $T+1$, ${ c }_{ T+1 }$ el cual puede ser el mismo estado latente ${ c }_{ t }$ o puede ser uno diferente. En cada uno de los estados latentes de la membres\'{i}a estos son mutuamente excluyentes y exhaustivos, es decir, los individuos pertenecen solo a un estado latente en cada tiempo. Por consiguiente se cumple:
			
			\begin{equation}
				\sum_{ { c }_{ T+1 }=1 }^{ C }{ { \tau  }_{ { c }_{ t+1 }|{ c }_{ t } }=1 } , \label{2.21}
			\end{equation}
			
			En otras palabras, cada fila de la matriz de probabilidades de transici\'{o}n debe sumar 1
			
		%%
		\subsection{Restricci\'{o}n de par\'{a}metros en el an\'{a}lisis de transici\'{o}n latente}
			Algunas restricciones en los par\'{a}metros pueden ser usadas en el modelo LTA. Estas se fijan en la prevalencia de los estados latentes, en las probabilidades de respuesta al item, o en las probabilidades de transici\'{o}n en el modelo de transici\'{o}n latente. Dos tipos diferentes de restricci\'{o}n de par\'{a}metros son comunmente usados: los par\'{a}metros pueden ser fijos o restringidos. 
			Un par\'{a}metro que es fijo en cierto valor particular no es estimado. Antes de comenzar la estimaci\'{o}n, su valor debe ser especificado en el rango de 0 a 1. Este valor fijo est\'{a} fuera de los limites del procedimiento de estimaci\'{o}n. 
			Cuando los par\'{a}metros est\'{a}n restringidos, estos son ubicados en un conjunto equivalente junto con otros par\'{a}metros. La estimaci\'{o}n de todos los par\'{a}metros en un conjunto equivalente est\'{a} restringida a ser igual al mismo valor, el cual puede ser cualquier valor en el rango de 0 a 1. Un modelo simple de transici\'{o}n latente puede contener cualquier combinaci\'{o}n de par\'{a}metros fijos y restringidos.          
			Debido a que los par\'{a}metros fijos no se estiman, estos no contribuyen al n\'{u}mero total de par\'{a}metros a estimar. El conjunto equivalente cuenta como un \'{u}nico parametro de estimaci\'{o}n, independientemente de cuantos par\'{a}metros conforman el conjunto. 
			
		%%
		\subsection{Restricciones en las probabilidades de respuesta al \'{i}tem}
			Es una buena idea restringir las probabilidades de respuesta al item en el modelo LTA para que estas sean iguales a trav\'{e}s del tiempo donde sea posible y razonable hacerlo, es decir, donde quiera una medida de invarianza a trav\'{e}s del tiempo puede ser supuesta. Existen razones conceptuales y practicas para hacer esto.
			La raz\'{o}n conceptual es que estos modelos de transici\'{o}n latente son mucho m\'{a}s faciles de interpretar si las probabilidades de respuesta al item son id\'{e}nticas a trav\'{e}s del tiempo. Se ha discutido mucho acerca de la medida de invarianza a trav\'{e}s de los grupos y tambi\'{e}n como probar hip\'{o}tesis acerca de la medida de invarianza. Si la probabilidades de respuesta al item son iguales a lo largo de los grupos entonces la interpretaci\'{o}n de las clases latentes es identica a lo largo de los grupos tambi\'{e}n. Esto significa que cualquier diferencia en los grupos observados en la prevalencia de las clases latentes pueden ser interpretadas simplemente como diferencias cuantitativas; ciertas clases latentes ser\'{a}n m\'{a}s grandes en algunos grupos que en otros. Por otro lado, si las probabilidades de respuesta al item no son iguales a lo largo de los grupos, entonces el significado de las clases latentes varia a lo largo de los grupos.
			Si esto sucede las comparaciones en los grupos de las prevalencias en las clases latentes llegar a ser menos sencillas. Cuando se interpreta las diferencias en la prevalencia de clases latentes se hace necesario tener en cuenta cualquier diferencia en el significado de las clases latentes en el mismo tiempo. Desde luego, dependiendo de las preguntas de investigaci\'{o}n, algunas veces las diferencias cualitativas pueden ser interesantes en si mismas.
			En el mismo sentido, la matriz de probabilidades de transici\'{o}n en el LTA es f\'{a}cil de interpretar si hay medidas equivalentes a trav\'{e}s de los tiempos. Si las probabilidades de respuesta al item son id\'{e}nticas a trav\'{e}s del tiempo, el significado de los estados latentes permanece constante a trav\'{e}s del tiempo. Esto significa por ejemplo, que un elemento en la diagonal de la matriz de probabilidades de transici\'{o}n refleja la probabilidad de la membresia en el estado latente $s$ en el tiempo $t+1$ condicionada a la membresia en el mismo estado latente $s$ en el tiempo $t$. Para extender de que las probabilidades de respuesta al item correspondientes al estado latente $s$ cambie a trav\'{e}s del tiempo, el significado del estado latente $s$ cambiar\'{a}. Luego, ya no es tan claro como interpretar esta probabilidad de transici\'{o}n, porque junto con la interpretaci\'{o}n cuantitativa tambi\'{e}n cambia a trav\'{e}s del tiempo la membresia del estado latente. Es necesario entonces interpretar el cambio a trav\'{e}s del tiempo en el sentido de las clases latentes. Sin embargo, dependiendo de las preguntas de investigaci\'{o}n este cambio cualitativo puede ser muy interesante, particularmente si es significativo para su desarrollo.           
			
			La raz\'{o}n principal para restringir las probabilidades de respuesta al item a que sean iguales a trav\'{e}s del tiempo es para ayudar a estabilizar la estimaci\'{o}n y mejorar el problema de identificabilidad. En los modelos de transici\'{o}n latente pueden haber un gran n\'{u}mero de probabilidades de respuesta al item, particularmente en modelos con m\'{a}s de dos mediciones en el tiempo. Imponer restricciones a los par\'{a}metros en el tiempo puede reducir considerablemente el n\'{u}mero de par\'{a}metros a estimar.
		
	\section{Modelos de diagn\'{o}stico cognitivo}
		Los modelos de diagn\'{o}stico cognitivo han sido desarrollados para estudiar los procesos cognitivos subyacentes al aprendizaje y para proporcionar informaci\'{o}n formativa acerca de cada estudiante. En t\'{e}rminos m\'{a}s t\'{e}cnicos, el diagn\'{o}stico cognitivo educacional es una evaluaci\'{o}n usada para identificar los atributos cognitivos que los estudiantes necesitan tener para lograr un dominio en particular y para clasificarlos dentro de grupos de diagn\'{o}stico basados en los atributos que ellos poseen (Rupp et al.,2010). Un diagn\'{o}stico cognitivo permite conocer un perfil detallado de habilidades dominadas o no dominadas para cada examinado m\'{a}s alla de darnos un puntaje general como lo hace la teor\'{i}a de respuesta al item. Adem\'{a}s cada participante es clasificado dentro de una clase cognitiva en la cual todos los individuos tienen el mismo patr\'{o}n en el dominio de los atributos.
		
		Antes de usar la evaluaci\'{o}n de diagn\'{o}stico cognitivo, es necesario primero identificar las variables latentes que ser\'{a}n medidas en una evaluaci\'{o}n en particular, asi como los items que buscar\'{a}n medir estas variables. Estas variables latentes han recibido diversos nombres en la literatura como por ejemplo: habilidades, atributos, rasgos latentes o perfiles latentes. Para especificar que variables son medidas con que items, la evaluaci\'{o}n cognitiva parte de la llamada matriz $Q$:
		
		\begin{equation} 
			Q = \left[ \begin{array}{cccc} 
				q_{11} & q_{12} & \ldots & q_{1K} \\  
				q_{21} & q_{22} & \ldots & q_{2K} \\    
				\vdots & \vdots & \ddots & \vdots \\
				q_{J1} & q_{J2} & \ldots & q_{JK} \\    
			\end{array} \right] \label{2.22}
		\end{equation}
		
		Esta matriz define la relaci\'{o}n entre todos los $J$ items y $K$ atributos, nombre que en adelante usaremos para representar a las variables latentes. Especificamente, la matriz Q es una tabla de items por atributo que especifica si cada atributo es medido o no por un item a trav\'{e}s de valores binarios 1s y 0s.\\
		Se realiza un test constru\'{i}do por expertos en el \'{a}rea de inter\'{e}s quienes definen una matriz $Q$ que indica los atributos requeridos para responder cada uno de los $J$ items del test. As\'{i}, ${q}_{ jk }$ toma el valor de 1 si el atributo $k$ es relevante para responder correctamente el item $j$ y 0 en caso contrario.
		Por ejemplo, para un test con 10 items que evalua la presencia de 4 atributos, una matriz $Q$ podr\'{i}a estar dada por:
		
		\begin{equation} 
			Q = \left[ \begin{array}{cccc} 
				0 & 1 & 0 & 1 \\  
				1 & 0 & 0 & 0 \\
				1 & 1 & 1 & 0 \\  
				0 & 1 & 0 & 0 \\
				1 & 0 & 0 & 0 \\  
				0 & 1 & 1 & 1 \\
				1 & 0 & 1 & 1 \\  
				0 & 0 & 1 & 0 \\
				0 & 0 & 1 & 1 \\  
				1 & 1 & 0 & 0 \\
			\end{array} \right] \label{2.23}
		\end{equation}
		
		La fila 1 indica que para responder correctamente el item 1, es necesario que el individuo posea el segundo y cuarto atributo, m\'{a}s no el primero y tercero.
		
		La creaci\'{o}n de esta matriz es realizada comunmente por expertos en la materia, bas\'{a}ndose en juicios te\'{o}ricos. Cualquier error de especificaci\'{o}n en la matriz Q puede causar errores en la clasificaci\'{o}n de los examinados al considerarlos en los grupos de diagn\'{o}stico err\'{o}neos. Para minimizar este error de clasificaci\'{o}n debido a la falsa especificaci\'{o}n de la matriz $Q$ y para mejorar la evaluaci\'{o}n varios m\'{e}todos de especificaci\'{o}n han sido propuestos(Chiu, 2013, Close ,2012; J. Liu, Xu and Ying, 2012; Xu, 2013).\\
		Diferentes modelos de diagn\'{o}stico cognitivo aparecen en la literatura, los cuales pueden clasificarse como compensatorios y no compensatorios. Los CDMs compensatorios permiten que una habilidad requerida para responder un item pueda ser compensada con otra habilidad adicional. Mientras, que los modelos no compensatorios requieren que cada una de las habilidades est\'{e}n presentes para producir una respuesta correcta al item.
		Como un ejemplo de modelo no compensatorio tenemos al modelo DINA. De otro lado, como ejemplo de un modelo compensatorio podemos considerar el modelo DINO.
		Seguidamente, describiremos brevemente estos modelos y su generalizaci\'{o}n.
		
		\subsection{El modelo DINA}
			El modelo DINA(de la Torre y Douglas, 2004; Haertel, 1989; Junker and Sijtsma, 2001; Macready and Dayton, 1977) es un CDM no compensatorio que asume que la falta de un atributo  no puede ser compensada por la existencia de otro atributo. Esta restricci\'{o}n significa que para responder correctamente a un item, un examinado necesita en teor\'{i}a dominar todos los atributos requeridos para ese item. La principal restricci\'{o}n en el modelo DINA es que este no hace distinciones entre los evaluados que no dominaron uno o m\'{a}s atributos de los requeridos.
			
			El modelo DINA estima la relaci\'{o}n entre $N$ individuos y las $K$ habilidades bas\'{a}ndose en el patr\'{o}n de respuestas de los individuos a los items, el cual es expresado como una matriz de unos y ceros, donde 1 se asigna cuando el individuo responde correctamente al item j y 0 en caso contrario. Esta relaci\'{o}n se resume en una matriz $N\times K$, llamada $A$ y que se define por:
			
			\begin{equation} 
				A = \left[ \begin{array}{cccc} 
					\alpha_{11} & \alpha_{12} & \ldots & \alpha_{1k} \\  
					\alpha_{21} & \alpha_{22} & \ldots & \alpha_{2n} \\    
					\vdots & \vdots & \ddots & \vdots \\
					\alpha_{N1} & \alpha_{N2} & \ldots & \alpha_{NK} \\    
				\end{array} \right], \label{2.24}
			\end{equation}
			
			la cual posee al igual que $Q$ una estructura binaria, es decir, con elementos 0 y 1, donde cada ${ \alpha  }_{ ik }$ se define como:
			
			\begin{equation}
				\alpha_{i k}=\left\{\begin{array}{cl}
							1 & ,\mbox{si el individuo $i$ domina la habilidad $k$ }\\
							0 & ,\mbox{en caso contrario}
						\end{array}\right. \label{2.25}
			\end{equation}
			
			A diferencia de la matriz $Q$, las entradas en la matriz A son latentes; es decir, no observables.\\    
			El modelo DINA del acr\'{o}nimo en ingl\'{e}s (Deterministic Input Noisy AND gate) es de naturaleza conjuntiva, es decir, que para que un individuo $i$ responda correctamente al item $j$ necesitar\'{a} de todas las habilidades que conciernen a tal item, el proceso tendr\'{a} como salida 1 si el individuo posee todas las habilidades requeridas para responder correctamente al item o 0 en caso contrario. Ello hace que sea necesario definir las siguientes variables dicot\'{o}micas:
			
			\begin{equation}
				\eta_{i j}=\left\{\begin{array}{cl}
							1	& ,\mbox{ si el individuo } i \text { domina todas las habilidades que } \\
								& \mbox{se necesitan para responder correctamente el item } j \\
							 {0} & ,\mbox{ en caso contrario}
						\end{array}\right. \label{2.26}
			\end{equation}
			
			y se cumple que $\eta_{i j}=\prod_{k=1}^{K} \alpha_{i k}^{q_{j k}}$.\\
			
			Adem\'{a}s, la denominaci\'{o}n de ruido atribu\'{i}da al modelo se da pues, pudiera darse el caso que los individuos que dominan todas las habilidades necesarias para un item fallen y aquellos que no las dominan todas acierten.
			Esto introduce dos probabilidades de error(par\'{a}metros), los cuales al definirse la respuesta binaria del individuo $i$ a un item $j$ por $Y_{ij}$ vienen dadas por:
			
			\begin{itemize}
				\item El par\'{a}metro de desliz del item $j$ $({ s }_{ j })$ es definido como la probabilidad de responder incorrectamente a un item $j$ a pesar de que el examinado domine todos los atributos requeridos para hacerlo, es decir:
				
				\begin{equation}
					{s}_{j}=P({Y}_{ij}=0|{ \eta  }_{ ij }=1). \label{2.27}
				\end{equation}
				
				\item El par\'{a}metro de adivinaci\'{o}n $({g}_{j})$ es definido como la probabilidad de responder correctamente a un item $j$ a pesar de no dominar todas las habilidades requeridas para hacerlo:
				
				\begin{equation}
					g_{ j }=P({ Y }_{ ij }=1|{ \eta  }_{ ij }=0). \label{2.28}
				\end{equation}
			\end{itemize}	
				
			Adem\'{a}s de la estimaci\'{o}n de estos par\'{a}metros el modelo DINA requiere estimar las probabilidades ${ \pi  }_{ c }=P({ \boldsymbol{\alpha_{i} }}={\boldsymbol{\alpha_{c}})}$ de pertenencia de cada individuo $i$ a la clase o perfil latente $c$. En adelante denotaremos por $L$ a la variable aleatoria que nos indica la clase de pertenencia de un individuo.
			
			El modelo DINA define la probabilidad de que un individuo $i$, que pertenece a la clase $c$, responda correctamente a un item $j$ mediante:
			
			\begin{equation}
				P_{c j}=P\left(Y_{i j}=1 | \alpha_{i}=\alpha_{c}, \theta_{j}\right)=\left(1-s_{j}\right)^{\eta_{i j}} g_{j}^{1-\eta_{i j}}, \label{2.29}
			\end{equation}
			
			donde $\theta_{ j }=({ g }_{ j },{ s }_{ j })$ es un vector fila de las probabilidades de error del item ya definidos anteriormente. En general, asumiendo independencia condicional y usando la ecuaci\'{o}n anterior, la probabilidad de observar la respuesta $x_{i j} \in\{0,1\}$  en el individuo $i$, que pertenece a la clase $c$, viene dada por:
			
			\begin{equation}
				P({ Y }_{ ij }={ y }_{ ij }\quad|{ \alpha  }_{ i }={ \alpha}_{ c },{ \theta  }_{ j })=p_{ cj }^{ { y }_{ ij } }(1-p_{ cj })^{ 1-{ y }_{ ij }}, \label{2.30}
			\end{equation}
			
			donde ${P}_{cj}$ viene dada por la expresi\'{o}n (\ref{2.29}).
			Utilizando el teorema de probabilidad total y la ecuaci\'{o}n (\ref{2.30}), se deduce que la probabilidad de obtener por parte del individuo $i$ un patr\'{o}n de respuestas $\boldsymbol{y}=[y_{1}, y_{2}, \ldots, y_{J}]$ dado los par\'{a}metros de todos los items, que resumimos en un vector $\theta$, y las probabilidades de pertenencia a las clases latentes $\pi=\left(\pi_{1}, \pi_{2}, \dots, \pi_{C}\right)$, est\'{a} dada por:
			
			\begin{equation}
				P\left(\boldsymbol{Y}_{i}=\boldsymbol{y}| \boldsymbol{\alpha}_{i}, \boldsymbol{\pi}, \boldsymbol{\theta}\right)=\sum_{c=1}^{C} \pi_{c}\left[\prod_{j=1}^{J} P\left(Y_{ij}=y_{j} | \boldsymbol{\alpha}_{i}=\boldsymbol{\alpha}_{c}, \boldsymbol{\theta}_{j}\right)\right] \label{2.31}  	
			\end{equation}
			
			Con esto se obtiene finalmente la siguiente funci\'{o}n de verosimilitud de observar una muestra de respuestas de $N$ individuos a los $J$ items:
			
			\begin{equation}
				L(\boldsymbol{\alpha}, \boldsymbol{\pi}, \boldsymbol{\theta})=\prod_{i=1}^{N} P\left(\boldsymbol{Y}_{\boldsymbol{i}}=\boldsymbol{y}_{i} | \boldsymbol{\alpha}_{i}, \boldsymbol{\pi}, \boldsymbol{\theta}\right)=\prod_{i=1}^{N}\left\{\sum_{c=1}^{C} \pi_{c}\left[\prod_{j=1}^{J} P_{c j}^{y_{i j}}\left(1-P_{c j}\right)^{1-y_{i j}}\right]\right\}, \label{2.32}  	
			\end{equation}
			
			donde $\boldsymbol{\alpha}=\left(\boldsymbol{\alpha}_{1}, \boldsymbol{\alpha}_{2}, \ldots, \boldsymbol{\alpha}_{C}\right)$ y ${ \boldsymbol{y} }_{i}=\left( y_{ 11 },\ldots ,y_{ i_{J} } \right)$ denota al patr\'{o}n de respuestas observado para el examinado $i$.
			
	
		\subsection{El modelo DINO}
			El modelo DINO (del acr\'{o}nimo en ingl\'{e}s deterministic input, noisy or gate) es un modelo de diagnostico cognitivo compensatorio(Templin,2004; Templin, et al.,2006), porque asume que la falta de un atributo puede ser compensada por otro. En otras palabras, el dominar al menos un atributo compensa el deficit de no dominar todos los otros atributos medidos.
			De la misma manera que en el modelo DINA, los par\'{a}metros de adivinaci\'{o}n y desliz son estimados en los niveles de los items. El modelo DINO se caracteriza por emplear una regla de compensaci\'{o}n disyuntiva en la cual la presencia de al menos una medida del atributo garantiza una probabilidad alta de aprobaci\'{o}n del item.
			El modelo DINO estima la probabilidad de respuesta correcta al item $j$ por parte de un individuo $i$ que pertenece a la clase latente $c$ de la siguiente manera:
			
			\begin{equation}
				P_{cj}=P\left(Y_{i j}=1 |{ \alpha  }_{ i }={ \alpha}_{ c },{ \theta  }_{ j }\right)=\left(1-s_{j}\right)^{\omega_{i j}} g_{j}^{\left(1-\omega_{i j}\right)}, \label{2.33}  	
			\end{equation}
			
			donde:
			
			\begin{equation}
				\omega_{ ij }=1-\prod_{k=1}^{K}\left(1-\alpha_{i k}\right)^{q_{j k}}.  \label{2.34}  	
			\end{equation}
			
			En caso de que el atributo $k$ no sea medido por el item $j$, $q_{j k}$ tomar\'{a} el valor de 0 y en consecuencia el valor de $1-\alpha_{i k}$ no importar\'{i}a. Por otra parte, si el atributo $k$ es medido por el item $j$,  $q_{j k}$ tomar\'{a} el valor de 1, y en consecuencia  $1-\alpha_{i k}$ se toma en cuenta para el valor final que ${ \omega  }_{ ij }$ tome. Si el respondiente en la clase latente $i$ domina el atributo $k$,  $\alpha_{i k}$ toma el valor de 1, y de este modo $1-\alpha_{i k}$ ser\'{a} igual a 0. Sin embargo, si el respondiente en la clase latente $i$ no domina el atributo $k$, entonces  $1-\alpha_{i k}$ es 1. 
			Debido a que la ocurrencia de ${ \omega  }_{ ij }=1$ depende de la existencia de al menos un 0 en el t\'{e}rmino de multiplicaci\'{o}n, el dominio de al menos un atributo aumenta significativamente la probabilidad de acertar el item.
			El modelo DINO es usado cuando solo un atributo se necesita para dominar la habilidad a diferencia de otros que necesitan m\'{a}s de un atributo(Rupp et al., 2010). Los par\'{a}metros de adivinaci\'{o}n ${ g }_{ j }$ y desliz ${ s }_{ j }$ en el modelo DINO se definen de la misma manera que en el modelo DINA.      
			
			\subsection{El modelo G-DINA} \label{sec:GDINA}
				Como se coment\'{o} anteriormente los modelos DINA y DINO dividen para cada item $j$ a las personas en dos grupos. En el modelo DINA por ejemplo, un primer grupo est\'{a} compuesto por los individuos que presentan todos los atributos requeridos para dar una respuesta correcta a este item, y el segundo grupo formado por el resto de individuos.
				El modelo G-DINA (del acr\'{o}nimo en ingl\'{e}s Generalized Deterministic Input Noisy and gate) cuestiona la suposici\'{o}n de igual probabilidad de responder correctamente para los individuos del segundo grupo y propone una generalizaci\'{o}n. As\'{i}, en lugar de formar solo dos grupos, el modelo G-DINA forma ${ 2 }^{ { K }_{ j }^{ * } }$ grupos, donde $K_{j}^{*}$ es el n\'{u}mero de atributos requeridos para el item $j$. Por ejemplo cuando $K_{j}^{ * } =2$ , en lugar de 2, se crean 4 grupos latentes, donde cada uno de ellos pueden tener diferentes probabilidades de \'{e}xito.
				
				El modelo generalizado de entrada determin\'{i}stica con ruido y salida (G-DINA) parte de un contexto en el que $N$ individuos denotados por $i = 1,2,\ldots ,N$, son examinados mediante un test con $J$ items denotados por $j = 1, 2,\ldots ,J$, cada uno de los cuales requieren la presencia de ${ K }^{*}_{ j }$ atributos para ser respondidos correctamente, donde ${K}_{j}^{*}\le K$, siendo $K$ la totalidad de atributos evaluados por el test.
				
				De los resultados del test obtenemos una matriz observable binaria ${y}=\left[ {y}_{ij} \right]$  de orden $N\times J$ que representa las respuestas de los $N$ entrevistados a los $J$ items. De tal manera, que ${x}_{ij}$ toma el valor 1 si el individuo respondi\'{o} correctamente al item $j$ y 0 en caso contrario.
				
				\begin{equation} 
					y = \left[ \begin{array}{cccc} 
					y_{11} & y_{12} & \ldots & y_{1J} \\  
					y_{21} & y_{22} & \ldots & y_{2J} \\    
					\vdots & \vdots & \ddots & \vdots \\
					y_{N1} & y_{N2} & \ldots & y_{NJ} \\    
					\end{array} \right].\label{2.35}
				\end{equation}
				
				Para cada individuo $i$ se asume que existe un vector latente ${ \alpha  }_{ i }=\left[ { \alpha  }_{ i1 },{ \alpha  }_{ i2 },\ldots ,{ \alpha  }_{ ik } \right]$ tal que $\alpha_{i k}=1$ si el individuo $i$ posee el atributo $k$, y $\alpha_{i k}=0$ en caso contrario. Estos, que llamaremos perfiles latentes, son desconocidos y deben ser estimados.
				Para cada item $j$, se separan a los individuos en  $2^{k_{j}^{*}}$ grupos latentes ($l$), donde ${K}_{ j }^{ * }=\sum _{ k=1 }^{ K }{ { q }_{ jk }}$ representa el n\'{u}mero de atributos requeridos para el item $j$. As\'{i}, cada individuo pertenecer\'{a} a solo una de estas clases. La presencia o ausencia de los atributos que no se requieren para responder correctamente este item, no afecta la pertenencia de un individuo a una u otra clase.
				
				Retornando al ejemplo en la matriz $Q$ dada en (2.22), vemos que el item 3 solo requiere de la presencia de 3 atributos por lo que para este item se forman ${2}^{3}=8$ grupos latentes.\
				
				Con el fin de simplificar la notaci\'{o}n y sin perder generalidad, consideraremos siempre que son los primeros ${K}_{ j }^{*}$ atributos los requeridos para responder correctamente el item $j$, y que correspondientemente ${a}_{ lj }^{ * }$ es el vector binario de dimensi\'{o}n ${ K }_{ j }^{ * }$ que contiene 1s solo si un individuo cualquiera de la clase $l$ posee tales componentes para responder correctamente el item $j$. Por ejemplo, el item 10 en la matriz solo requiere dos atributos para ser respondido correctamente, por lo que el vector ${ a }_{ lj }$ toma el valor de ${ a }_{ lj }^{ * }=\left( { a }_{ lj1 },{ a }_{ lj2 } \right)$, en lugar del vector completo $\left( { a }_{ lj1 },{ a }_{ lj2 },{ a }_{ lj3 },{ a }_{ lj4 } \right)$
				
				En modelos m\'{a}s sencillos como el DINA, carecer de un atributo requerido para determinado item, es lo mismo que carecer de todos los atributos requeridos. Sin embargo, esto podr\'{i}a no ser siempre cierto, ya que un individuo que posee alguno de los ${K }_{ j }^{ * }$ atributos requeridos para el item $j$, podr\'{i}a tener mayor probabilidad de responder correctamente que aquel que no tiene ninguno. El modelo G-DINA relaja esta hip\'{o}tesis de igual probabilidad de \'{e}xito.
				
				Por esta raz\'{o}n es importante establecer una relaci\'{o}n entre los vectores $\alpha_{l j}^{*}$ y $\alpha_{l^{\prime} j}^{*}$ que denotan a los estados de conocimiento o perfiles de atributos de 2 sujetos en las clases $l$ y ${ l }^{ \prime  }$. As\'{i}, para esta investigaci\'{o}n, diremos que $\alpha_{l j}^{*}<\alpha_{l^{\prime} j}^{*}$, si $\alpha_{l j}^{*}$ posee menos atributos de los requeridos para el item $j$ que $\alpha_{l^{\prime} j}^{*}$, es decir, $\sum_{k=1}^{K_{j}^{*}} \alpha_{l j k}^{*}<\sum_{k=1}^{K_{j}^{*}} \alpha_{l^{\prime} j k}^{*}$\\ 
				En el ejemplo de la matriz $Q$, para el item 3, que requiere la presencia de tres atributos, podemos afirmar que el vector de estado $a_{ l3 }^{ * }=(0,0,1)$ es menor que el vector $a_{l^{\prime}3 }^{ * }=(1,1,0)$, ya que posee una menor cantidad de atributos requeridos.
				
				En adelante denotaremos a la probabilidad de que los entrevistados con perfil de atributos  $\alpha_{l j}^{*}$ respondan el item $j$ correctamente por:
				$P\left(X_{j}=1 | \alpha_{l j}^{*}\right)=P\left(\alpha_{l j}^{*}\right)$, siendo $X_{j}$ la variable aleatoria que denota la respuesta de un entrevistado al item $j$. Como es natural estas probabilidades deber\'{a}n de satisfacer que $P\left(\alpha_{l j}^{*}\right) \leq P\left(\alpha_{l^{\prime} j}^{*}\right)$ cuando $\alpha_{l j}^{*}<\alpha_{l^{\prime} j}^{*}$ . El modelo G-DINA plantea para estas probabilidades la siguiente ecuaci\'{o}n:
				
				\begin{equation}
					P\left(\alpha_{l j}^{*}\right)=\delta_{j 0}+\sum_{k=1}^{K_{j}^{*}} \delta_{j k} \alpha_{l k}+\sum_{k=1}^{K_{j}^{*}-1} \sum_{k^{\prime}=k+1}^{K_{j}^{*}} \delta_{j k k^{\prime} \alpha_{l k} \alpha_{l k^{\prime}}}+\ldots+\delta_{j 12 \ldots K_{j}^{*}} \prod_{k=1}^{K_{j}^{*}} \alpha_{l k}, \label{2.36}
				\end{equation}
				
				donde $\delta_{j}=\left(\delta_{j 0}, \delta_{j 1}, \ldots, \delta_{j K_{j}^{*}}, \delta_{j 12}, \ldots, \delta_{j 12 \ldots K_{j}^{*}}\right)$, representa un vector de par\'{a}metros a estimarse para el item $j$, $\delta_{j0}$ es un par\'{a}metro de intercepto (concretamente la probabilidad de que los encuestados respondan correctamente al item $j$ cuando no posean ninguno de los atributos requeridos para responder satisfactoriamente este item), ${ \delta  }_{ jk }$ representa el efecto principal debido al atributo $k$ (el incremento marginal en la probabilidad de responder correctamente al item $j$ como resultado de la presencia del atributo $k$); ${\delta}_{jkkj}$ representa el efecto de segundo orden, (el incremento marginal en la probabilidad de obtener una respuesta correcta debido a la presencia de los atributos $k$ y ${ k }_{ j }$); y as\'{i} sucesivamente hasta ${ \delta  }_{ j12\ldots { { k }_{ j }^{ * } } }$ que representa el cambio en la probabilidad de obtener una respuesta correcta debido a la presencia de todos los atributos requeridos.
				Uno de los aspectos que resaltan la importancia del modelo G-DINA, es que adem\'{a}s de medir la contribuci\'{o}n marginal de poseer un atributo particular en las probabilidades de responder correctamente un item, puede medir tambi\'{e}n (si existiera) el efecto conjunto de dos o m\'{a}s atributos en ella.
				
				     
				
				
				
				
		
		

	
	%Modelo
	
\chapter{Modelos de transici\'{o}n de clases latentes cognitivas} \label{cap:modelo}
	Seg\'{u}n \cite{kaya2017assessing} el modelo de transici\'{o}n de clases latentes cognitivas permite investigar hip\'{o}tesis acerca del efecto de una intervenci\'{o}n mediante el cambio en las probabilidades de transici\'{o}n antes y despu\'{e}s de esta con relaci\'{o}n al estado cognitivo. Esta clase de modelos es una excelente manera de estudiar cambios en el desarrollo y crecimiento, tales como las transiciones entre los estados descritos en la teoria de Piaget, seg\'{u}n lo citado por \cite{li2016latent}.
	Suponga que observamos $J$ variables categ\'{o}ricas manifiestas polit\'{o}micas $Y_1, Y_2, ..., Y_j$, donde $Y_j$ tiene $R_j$ posibles resultados. Se asume que estas variables deben ser la respuesta de un individuo a una evaluaci\'{o}n con $J$ items. El modelo de transici\'{o}n de clases latentes con solo dos puntos en el tiempo ($T=2$), expresa que la distribuci\'{o}n de probabilidad conjunta de las respuestas $Y_{1} = [Y_{1,1},Y_{2,1}, ... Y_{J,1}]$, $Y_{2} = [Y_{1,2},Y_{2,2}, ... Y_{J,2}]$  de cualquier individuo, donde $Y_{j,t}$ denota a su respuesta al item $j$ en el tiempo $t$, viene dada por:

	\begin{align}
		P&\left[\mathbf{Y}_{1} =  \mathbf{y}_{1}, \mathbf{Y}_{2} =\mathbf{y}_{2}\right]  =  \notag \\
		& = \sum _{ { c }_{ 1 }=1 }^{ C }{ \sum _{ { c }_{ 2 }=1 }^{ C }{ P\left( { Y }_{ 1 }=\mathbf{ y }_{ 1 },{ Y }_{ 2 }=\mathbf{ y }_{ 2 }|{ C }_{ 1 }={ c }_{ 1 },{ C }_{ 2 }={ c }_{ 2 } \right) P\left( { C }_{ 2 }={ c }_{ 2 }|{ C }_{ 1 }={ c }_{ 1 } \right) P\left( { C }_{ 1 }={ c }_{ 1 } \right)  }  } \label{3.1} \\
		& =\sum _{ { c }_{ 1 }=1 }^{ C } \sum _{ { c }_{ 2 }=1 }^{ C }{ { \delta  }_{ { c }_{ 1 } }{ \tau  }_{ { c }_{ 2 }|{ c }_{ 1 } }P\left( { Y }_{ 1 }=\mathbf{y}_{1}|{ C }_{ 1 }={ c }_{ 1 } \right) P\left( { Y }_{ 2 }=\mathbf{y}_{2}|{ C }_{ 2 }={ c }_{ 2 } \right)} \\
		\end{align}
		
	\begin{equation}
	=\sum _{ { c }_{ 1 }=1 }^{ C } \sum _{ { c }_{ 2 }=1 }^{ C }{ { \delta  }_{ { c }_{ 1 } }{ \tau  }_{ { c }_{ 2 }|{ c }_{ 1 } }\prod _{ j=1 }^{ J }{ \prod _{ { r }_{ j,1 }=1 }^{ { R }_{ j } }{ { \rho  }_{ j,{ r }_{ j,1 }|{ c }_{ 1 } }^{ I\left( { y }_{ j,1 }={ r }_{ j,1 } \right)  } }  }  } \prod _{ j=1 }^{ J }{ \prod _{ { r }_{ j,2 }=1 }^{ { R }_{ j } }{ \rho _{ j,{ r }_{ j,2 }|{ c }_{ 2 } }^{ I\left( { y }_{ j,2 }={ r }_{ j,2 } \right)  } }  } . \hspace{3.2cm}\label{3.1}
	\end{equation}
	
	\begin{equation}
		=\sum _{ {c}_{1}=1 }^{C}{\sum _{ {c}_{2}=1 }^{C}{\delta _{c_1} \tau_{c_2|c_1} }}\prod _{ t=1 }^{ 2 }{  \prod_{ j=1 }^{J}{\prod_{ r_{j,t}=1 }^{R_j}{\rho_{j,r_{j,t}|c_t}^{I(y_{j,t}=r_{j,t})}  } }    } , \label{3.1}
	\end{equation}
	
	donde
	
	\begin{equation}
		\sum _{ c_{1} =1 }^{C}{\delta _{c_{1}} } =1, \sum_{c_{2}=1}^{C_{L}}{\tau_{c_{2} | c_{1}}}=1, \sum _{ r_{j,t}=1 }^{R_j}{\rho_{j,r_{j,t}|c_t} }  =1  ,\label{3.7}
	\end{equation}
$\delta _{c1}$ la probabilidad de pertenencia del individuo a la clase $c_1$ en el tiempo $1$, $\tau_{c2|c1}$ es la probabilidad de transici\'{o}n de que el individuo cambie a una clase $c_2$ en el tiempo $2$ dado que \'{e}l ha pertenecido a una clase $c_1$ en el tiempo $1$, $I(y_{j,t}=r_{j,t})$ es una variable indicadora que es igual a $1$ si la respuesta al item $j$ en el tiempo $t$ es  $r_{j,t}$ y $0$ en caso contrario, y $\rho_{j,r_{j,t}|c_t}$, es la probabilidad de que la respuesta del individuo al item $j$ en el tiempo $t$ sea $r_{j,t}$ condicionado a su membresia a la clase latente $c_t$.\\
	La consideraci\'{o}n de un modelo con m\'{a}s de dos puntos en el tiempo es inmediata pero implica el incremento del n\'{u}mero de par\'{a}metros, no s\'{o}lo en t\'{e}rminos de las probabilidades de transici\'{o}n, sino tambi\'{e}n en los par\'{a}metros dentro de las probabilidades de respuesta al item del modelo para $\rho_{j,r_{j,t}|c_t}$. Este modelo se presenta a continuaci\'{o}n:
	
	\begin{equation}
		P\left[\mathbf{Y}=\mathbf{y}\right]=\sum_{c_{1}=1}^{C} \ldots \sum_{c_{T}=1}^{C} \delta_{c_{1}} \tau_{c_{2}|c_{1}}^{(1)}\ldots \tau_{c_{T} | c_{T-1}}^{(T)} \prod_{t=1}^{T}{ \prod _{ j=1 }^{ J }{ \prod _{ { r }_{ j,t }=1 }^{ { R }_{ j } }{ { \rho  }_{ j,{ r }_{ j,t }|{ c }_{ t } }^{ I({ y }_{ j,t }={ r }_{ j,t }) } }  }  }  \label{3.8}
,
\end{equation}
donde $\tau_{{a}|{b}}^{T}$ representa la probabilidad de transici\'{o}n de que un individuo en la clase $b$ en el tiempo $t-1$ pase a la clase $a$ en el tiempo $t$\\
	En este cap\'{i}tulo detallaremos los modelos longitudinales DINA y DINO para clasificar a los examinados con respecto al dominio de atributos en m\'{u}ltiples per\'{i}odos de medici\'{o}n. Estos modelos clasifican a los examinados en los estados latentes a trav\'{e}s de mediciones consecutivas y estiman las probabilidades de transici\'{o}n entre clases de los examinados desde el tiempo $1$ hasta el tiempo $T$.\\
	El modelo de transici\'{o}n de clases latentes cognitivas DINA que denotaremos por LTA-DINA, estima la probabilidad de que un individuo presente un patrón particular de respuestas como una funci\'{o}n de sus probabilidades de membres\'{i}a a cada clase cognitiva en el tiempo 1, las probabilidades de transici\'{o}n entre las clases cognitivas y las probabilidades de observar una respuesta en cada punto en el tiempo condicionado a su clase cognitiva de membres\'{i}a. El modelo LTA-DINA integra el modelo LTA con el DINA a trav\'{e}s de:
	
	\begin{align}
		P&\left[\mathbf{Y}_{1} =  \mathbf{y}_{1},\mathbf{Y}_{2} =\mathbf{y}_{2},...,\mathbf{Y}_{T} =\mathbf{y}_{T}\right] =  
\notag \\
		& = \sum_{c_{1}=1}^{C} \ldots \sum_{c_{T}=1}^{C} \delta_{c_{1}} \tau_{c_{2} | c_{1}} \ldots \tau_{c_{T} | c_{T-1}} \prod_{t=1}^{T} \prod_{j=1}^{J}  \left( 1-{ P }_{ { c }_{ { t }_{ j } } } \right) ^{ I({ y }_{ j,t }=0) }_{  }{ P }_{ { c }_{ t }j }^{ I({ y }_{ j,t }=1) }, 
\label{3.9}
\end{align}
donde ${ P }_{ { c }_{ t }j }=(1-{ s }_{ j })^{ { \eta  }_{ { c }_{ t }j } }{ { g }_{ j }^{ (1-{ \eta  }_{ { c }_{ t }j }) } }$ es la probabilidad dada en (2.30) y ${\eta }_{ { c }_{ t }j }$ es el estado del conocimiento de los individuos en la clase $c_{t}$ para el item $j$\\
De la misma manera, podemos definir el modelo de transici\'{o}n DINO, que lo denotaremos por LTA-DINO mediante:
	\begin{equation}\label{3.10}
	P\left[\mathbf{Y}=\mathbf{y}\right]=\sum_{c_{1}=1}^{C} \ldots \sum_{c_{t}=1}^{C} \delta_{c_{1}} \tau_{c_{2} | c_{1}} \ldots \tau_{c_{t} | c_{t-1}} \prod_{t=1}^{T} \prod_{j=1}^{J}  \left( 1-{ P }_{ { c }_{ { t }_{ j } } } \right) ^{ I({ y }_{ j,t }=0) }_{  }{ P }_{ { c }_{ t }j }^{ I({ y }_{ j,t }=1) },
	\end{equation}
	donde ${ P }_{ { c }_{ t }j }=(1-{ s }_{ j })^{ { \omega }_{ { c }_{ t }j } }{ { g }_{ j }^{ (1-{ \omega }_{ { c }_{ t }j }) } }$ es la probabilidad dada en (2.34) y ${\omega}_{ { c }_{ t }j }$ es el estado de conocimiento para el modelo DINO, la cual lo diferencia del modelo DINA cuyo estado de conocimiento es ${\eta }_{ { c }_{ t }j }$.\\
	Adicionalmente, a la estimaci\'{o}n de estos par\'{a}metros, ser\'{a} de inter\'{e}s estimar las probabilidades de pertenencia aposteriori para determinar las clases para cada individuo en cada punto en el tiempo. Para el uso pr\'{a}ctico de los modelos longitudinales DINA y DINO, los par\'{a}metros de inter\'{e}s son probabilidades aposteriori y las probabilidades de transici\'{o}n. Estas brindan informaci\'{o}n individual y son usadas para evaluar el rendimiento de cada estudiante basados en su clase grupal estimada. 
	Por otro lado, las probabilidades de transici\'{o}n consideran las probabilidades de permanencia en la misma clase o el movimiento entre las clases de un tiempo $t$ a un tiempo $t+1$ para los examinados que est\'{a}n en la misma clase cognitiva, quienes nos brindan una idea general del \'{e}xito del tratamiento aplicado entre las 2 mediciones.           
	
	\section{Estimaci\'{o}n del modelo}	
El modelo de de transición latente en combinación con el modelo de diagnóstico cognitivo DINA(LTA-DINA)nos pemite analizar el comportamiento de los evaluados en las diferentes clases cognitivas a través de las probabilidades de transición. Para la estimación de este modelo desarrollaremos el método de máxima verosimilitud via el algoritmo de Esperanza-Maximización (EM). La implementación de este se encuentra en el paquete Mplus (Version 7; Muthén y Muthén, 2017).\\     
	Los par\'{a}metros en un modelo de transici\'{o}n latente incluyen la probabilidad de la membresia en el tiempo 1, las probabilidades de transici\'{o}n y los parámetros involucrados en la probabilidad de respuesta al item condicionadas a la clase latente.\
	
	La probabilidad conjunta de un individuo, que pertenece al conjunto de clases $\mathbf{c}=\left(c_{1}, \dots, c_{T}\right)$ en el tiempo, proporcione un cierto patrón de respuestas es:
	
	\begin{equation}
		P\left[\mathbf{ Y }_{ 1 }=\mathbf{ y }_{1},...,\mathbf{Y}_{ T }=\mathbf{ y }_{T}| {C}_={c} \right] =\left[ { \delta  }_{ c1 }\prod _{ t=2 }^{ T }{ { \tau  }_{ { c }_{ t }|{ c }_{ t-1 } }^{ (t) } }  \right] \times \left[ \prod _{ t=1 }^{ T }{ \prod _{ j=1 }^{ J }{ \prod _{ r_{j}=1 }^{ { R }_{ j } }{ { \rho  }_{ j,r_{j},t|{ c }_{ t } }^{ I({ y }_{ j,t }=r_{j,t}) } }  }  }  \right], \label{3.11}	
	\end{equation}	
donde ${ \delta  }_{ { c }_{ 1 } }=P\left[ { C }_{ 1 }={ c }_{ 1 } \right]$ ,${ \tau  }_{ { c }_{ t }|{c}_{ t-1 } }^{ (t) }=P\left[ { C }_{ t }={ c }_{ t }|{C}_{ t-1 }={ c }_{ t-1 } \right]$ y ${ \rho  }_{ j,r,j|{ c }_{ t } }=P\left[ { Y }_{ jt }=y_{j,t}|{ C }_{ t }={ c }_{ t } \right]$. Se asume que ${ Y }_{ 1t },...,{ Y }_{ Jt }$ son condicionalmente independientes dentro de cada clase ${ c }_{ t }$ para $t=1,...,T$. Este supuesto, llamado de independencia local nos permite realizar inferencias acerca de la variable de clase latente.(Lazarsfeld PF et al.,1968). Tambi\'{e}n se asume que la secuencia $\{ C_{t} \}$ donde ${C}_{t}$ denota a la clase de pertenencia de un individuo en el tiempo $t$, constituye una cadena de Markov de primer orden para $t=2,...,T$. En la ecuaci\'{o}n (3.10) solo la probabilidad marginal de la clase de la membresia en el tiempo de inicio $t=1$, ${ \delta  }_{ { c }_{ 1 } }$ es estimada; las probabilidades marginales de la clase de membres\'{i}a en el tiempo $t(\geqslant 2)$ no son estimadas directamente sin embargo, est\'{a}n en funci\'{o}n de otros par\'{a}metros. La prevalencia marginal de cada clase en el tiempo $t(\geqslant 2)$ es calculado mediante: 
	
	\begin{equation}
		\delta_{c_{t}}^{(t)}=P\left[C_{t}=c_{t}\right]=\sum_{c_{1}=1}^{C} \cdots \sum_{c_{t-1}=1}^{C} \delta_{c_{1}} \prod_{j=2}^{J} \tau_{c_{t} | c_{t-1}}^{(t)}. \label{3.12}     
	\end{equation}
	
	Entonces la función de verosimilitud para un individuo o distribución conjunta incondicional de su respuesta es:
	
	\begin{equation}
		P\left[\mathbf{Y}_{1}=\mathbf{y}_{1}, \ldots, \mathbf{Y}_{T}=\mathbf{y}_{T}\right]=\sum_{c_{1}=1}^{C} \cdots \sum_{c_{T}=1}^{C} \left[ { \delta  }_{ c1 }\prod _{ t=2 }^{ T }{ { \tau  }_{ { c }_{ t }|{ c }_{ t-1 } }^{ (t) } }  \right] \times \left[ \prod _{ t=1 }^{ T }{ \prod _{ j=1 }^{ J }{ \prod _{ r_{j}=1 }^{ { R }_{ j } }{ { \rho  }_{ j,r_{j},t|{ c }_{ t } }^{ I({ y }_{ j,t }=r_{j,t}) } }  }  }  \right] \label{3.13}
	\end{equation}
	
	Por simplicidad consideremos en este trabajo una muestra de $n$ individuos quienes responden a $J$ items binarios medidos en dos per\'{i}odos de tiempo. Consideremos aqui el modelo restringido de transici\'{o}n de clases latentes donde las probabilidades de respuesta al item son ajustadas para ser iguales durante el tiempo, aunque una extensi\'{o}n del modelo LTA sin restricciones es sencilla. En nuestro caso la funci\'{o}n de verosimilitud en (3.12) se reduce a:
	
	\begin{equation}
		P\left[\mathbf{Y}_{1}=\mathbf{y}_{1}, \mathbf{Y}_{2}=\mathbf{y}_{2}\right]=\sum _{ {c}_{1}=1 }^{C}{\sum _{ {c}_{2}=1 }^{C}{\delta _{c_1} \tau_{c_2|c_1} }}\prod _{ t=1 }^{ 2 }{  \prod_{ j=1 }^{J}{\prod_{ r_{j}=1 }^{2}{\rho_{j,r_{j}|c_t}^{I(y_{j,t}=r_{j,t})}}}},\label{3.14}
	\end{equation}
donde $\tau_{c_{2} | c_{1}}=P\left[C_{2}=c_{2} | C_{1}=c_{1}\right]$. 
	En (3.13), los par\'{a}metros libres son $\theta=\left(\delta, \tau_{1}, \dots, \tau_{C}, \rho_{1}, \dots, \rho_{C}\right)$ , donde $\delta=\left(\delta_{1}, \ldots, \delta_{C-1}\right), \tau_{c}=\left(\tau_{1 | c}, \ldots, \tau_{C-1 | C}\right)$ y $\boldsymbol{\rho}_{c}=\left(\rho_{11} | c, \cdots, \rho_{J 1 | c}\right)$ para $c=1, \dots, C$
	
	Bajo condiciones normales, los estimadores de m\'{a}xima verosimilitud para $\theta$ resuelven la funci\'{o}n score, $\partial \log \prod P\left[\mathbf{y}_{1}, \mathbf{y}_{2}\right] / \partial \theta=0$. Al igual que muchas mezclas finitas, los estimadores de m\'{a}xima verosimilitud para el LTA pueden ser estimados usando un algoritmo de optimización.
	
	La función de verosimilitud será maximizada usando el algoritmo EM (Esperanza-Maximización). El software Mplus (version 7) de (Muthen y Muthen, 2017) será usado para implementar la estimación del modelo LTA-DINA.	
	
	
	Para el paso E de esperanza, calculamos la probabilidad condicional de que cada individuo es miembro de la clase ${ c }_{ 1 }$ en el tiempo $t=1$ y de la clase ${ c }_{ 2 }$ en el tiempo $t=2$ dadas las respuestas al item $\mathbf{y}=$ $\left(\mathbf{y}_{1}, \mathbf{y}_{2}\right)$ y de las estimaciones actuales $\hat{\theta}$ para los par\'{a}metros,
	
	\begin{equation}
		\hat{\eta}{\left(c_{1}, c_{2}\right)}=P\left[C_{1}=c_{1}, C_{2}= c_{2} | \mathbf{y}_{1}, \mathbf{y}_{2}\right]=\frac { \delta _{ c_{ 1 } }\tau_{c_2|c_1}\prod _{ t } \prod _{ j } \prod _{ r_{j} } \rho_{j,r_{j} | c_{t}}^{I\left(y_{j,t}=r_{j,t}\right)} } { \sum _{ c_{ 1 } } \sum _{ c_{ 2 } } \delta _{ c_{ 1 } }\tau _{c_2|c_1}\prod _{ t } \prod _{ j } \prod _{ r_{j} } \rho_{j,r_{j} | c_{t}}^{I\left(y_{j,t}=r_{j,t}\right)}}. \label{3.15}   
	\end{equation}
	
	En el paso M de maximizaci\'{o}n, actualizamos las estimaciones de los par\'{a}metros por
	
	\begin{equation}
		\hat{\delta}_{c_{1}}=\frac{\hat{n}_{c_{1}}^{(1)}}{n}, \quad \hat{\tau}_{c_{2} | c_{1}}=\frac{\hat{n}_{\left(c_{1}, c_{2}\right)}}{\hat{n}_{c_{1}}^{(1)}}, \quad \hat{\rho_{}}_{j,r_{j}| c}=\frac{\hat{n}_{j,r_{j} | c}^{(1)}+\hat{n}_{j,r_{j} | c}^{(2)}}{\hat{n}_{c}^{(1)}+\hat{n}_{c}^{(2)}} , \label{3.16}
	\end{equation}
	
	donde:
	\begin{align}
		\hat{n}_{\left( c_{1}, c_{2} \right)} & = \sum_{j} \hat{\eta}_{ j\left(c_{1}, c_{2}\right)},   \hat{n}_{c_{1}}^{(1)} = \sum_{c_{2}} \hat{n}_{\left(c_{1},  c_{2}\right)},  \hat{n}_{c_{2}}^{(2)} = \sum_{c_{1}} \hat{n}_{ \left( c_{1}, c_{2} \right)} \nonumber \\
		\hat{n}_{j,r_{j} \vert c}^{(1)} & = \sum_{c_{2}} \sum_{j} I\left( y_{j,1} = r_{j} \right) \hat{\eta}_{ \left( c, c_{2} \right) }, \mbox{ y } \hat{n}_{j,r_{j} \vert c }^{(2)}=\sum_{c_{1}} \sum_{j} I\left(y_{j,2}=r_{j}\right) \hat{\eta}_{j\left(c_{1}, c\right)} \nonumber.
	\end{align} 
	
	La iteraci\'{o}n entre estos dos pasos produce una secuencia de par\'{a}metros estimados que convergen confiablemente a un m\'{a}ximo local o global de la funci\'{o}n de verosimilitud.
	Los desafios para la inferencia de m\'{a}xima verosimilitud en muestras peque\~{n}as para el modelo de transici\'{o}n de clases cognitivas se encuentran principalmente debido a los par\'{a}metros que han de ser estimados en los l\'{i}mites de un espacio param\'{e}trico(es decir, 0 o 1), lo que dificulta la obtenci\'{o}n adecuada de los errores est\'{a}ndar.
	Aunque las probabilidades de respuesta al item son cercanas a cero o a uno, \'{e}stas son altamente deseables desde una perspectiva de medici\'{o}n, cuando algunos de estos par\'{a}metros son estimados en el l\'{i}mite, es imposible obtener los errores est\'{a}ndar para la inversa de la matriz hessiana. 

El LTA-DINA es ajustado para los datos de la prueba FOC usando el paquete Mplus. Los fueron estimados con 15 puntos de integración, con un criterio de convergencia igual a $10^{-7}$ para el cambio absoluto en la log-verosimilitud y el numero máximo de iteraciones del EM fué de 100, utilizandose además la descomposición de Cholesky y una cuadratura adaptativa con un tipo de integración estándar y modelo de mixtura con el link logit.  	     
	
	
	%%-------------------------------------------------------------------------
	\section{Inferencia del modelo}
	A pesar de que las estimaciones por EM y MCMC han hecho al modelo LTA muy popular, la funci\'{o}n de verosimilitud para el modelo LTA puede tener caracter\'{i}sticas inusuales las cuales pueden afectar seriamente la inferencia. Por ejemplo, puede haber ciertas regiones continuas del espacio de par\'{a}metros para los cuales la log verosimilitud  es constante, lo que conduce a la indeterminaci\'{o}n de algunos par\'{a}metros. Para modelos LTA de dos clases se supone que la prevalencia de la clase $1$ es cero en el tiempo $1$ pero luego la clase $1$ emerge al tiempo $2$ 
	$(\delta_{1}=0 \text { y } \sum_{l_{1}} \delta_{l_{1}} \tau_{1 | l_{1}}>0)$.
	
	
	%%-------------------------------------------------------------------------
	\section{Criterios para la selecci\'{o}n del modelo}
	Varios estadisticos de ajuste relativos est\'{a}n disponibles para comparar dos o m\'{a}s modelos. En general, los indices de Criterio de informaci\'{o}n o estadisticos de prueba LR pueden ser usados dentro del contexto del modelamiento de variables latentes. El primero es apropiado para comparar modelos anidados o no anidados, mientras que el otro solo puede ser utilizado para modelos anidados.
	En este estudio nos basamos en los indices de criterio de informaci\'{o}n, especificamente en el AIC y BIC. A menudo m\'{u}ltiples indices se usan para comparar el ajuste de los datos al modelo entre un conjunto de modelos para los mismos datos cuando los estimadores de m\'{a}xima verosimilitud de los par\'{a}metros se han obtenido.
	Los indices de Criterio de informaci\'{o}n est\'{a}n basados en una forma de penalizaci\'{o}n de la funci\'{o}n de verosimilitud. Valores peque\~{n}os de estos indices indican un mejor ajuste. Sin embargo, diferentes indices podri\'{a}n seleccionar diferentes modelos de ajuste para los mismos datos debido a las diferencias en la funci\'{o}n de penalizaci\'{o}n aplicada a la verosimilitud. El indice AIC est\'{a} dado por:
	
	\begin{equation}
		\mathrm{AIC}=-2 \log L+2 p 
		\label{3.17}
	\end{equation}
	
	donde $p$ es el n\'{u}mero de par\'{a}metros estimados y $2p$ es usada como una penalizaci\'{o}n para la sobreparametrizaci\'{o}n y $L$ es la funci\'{o}n de verosimilitud. Para los modelos de diagn\'{o}stico cognitivo, $L$ representa el valor de la funci\'{o}n de verosimilutud marginal del modelo, y $p$ comprende el n\'{u}mero total de los par\'{a}metros de los itemes y los par\'{a}metros estructurales. Uno de los problemas con el AIC es que tiende a seleccionar modelos m\'{a}s complejos. La falta de penalizaci\'{o}n para el tama\~{n}o de muestra conduce a la inconsistencia en el desempe\~{n}o o actuaci\'{o}n del AIC y una tendencia a sobreestimar el correcto n\'{u}mero de clases(McLachlan \& Peel, 2000).
	Sin embargo, el AIC es correcto y efciente asint\'{o}ticamente, si el verdadero modelo no est\'{a} entre los modelos que se comparan. (Vrieze, 2012). 
	Para tener en cuenta el tama\~{n}o de la muestra, el BIC puede ser usado y viene dado por:         
	
	\begin{equation}
		\mathrm{BIC}=-2 \log L+p \ln (N)
		\label{3.18}
	\end{equation}
	
	donde L es la verosimilitud del modelo estimado con $p$ par\'{a}metros libres y $ln(N)$ es la funci\'{o}n logaritmo del tama\~{n}o de muestra total $N$. Como podemos observar en la ecuaci\'{o}n 3.18, la funci\'{o}n de penalizaci\'{o}n para el BIC est\'{a} basada en el n\'{u}mero de par\'{a}metros estimados asi como tambi\'{e}n en el tama\~{n}o de muestra. El BIC tiende a aplicar mayor penalizaci\'{o}n a la funci\'{o}n de verosimilitud que el AIC cuando modelos complejos son estimados. Como consecuencia, el BIC es m\'{a}s preferido a la hora de seleccionar modelos simples que el AIC debido a la inclusi\'{o}n del tama\~{n}o de muestra en la funci\'{o}n de penalizaci\'{o}n. La penalizaci\'{o}n del BIC con $N$ hace de la significaci\'{o}n estadistica m\'{a}s y m\'{a}s dificil de lograr a su vez que el tama\~{n}o de muestra aumenta(Vrieze, 2012, p. 233).
	Como sabemos, a diferencia del AIC, el BIC est\'{a} hecho para ser asint\'{o}ticamente consistente(Shao, 1997), lo que significa que a medida que el tama\~{n}o de muestra aumenta, el BIC tiende a seleccionar el n\'{u}mero correcto de clases mixtas consistentemente si el verdadero modelo est\'{a} entre los modelos que son comparados. Por el contrario, a medida que el tama\~{n}o de muestra aumenta, el AIC tiende a seleccionar un modelo m\'{a}s complejo aun cuando el verdadero modelo est\'{e} dentro de los candidatos. Un estudio de Shao acerca de la selecci\'{o}n de indices de modelos, muestra que la utilidad del AIC y el BIC dependen b\'{a}sicamente de la estructura del modelo. El BIC se espera que actue mejor cuando el modelo verdadero tiene una estructura simple, y el AIC se espera que actue mejor cuando el modelo verdadero tiene una estructura compleja.
	Adicionalmente, con respecto al modelo verdadero, la utilidad del AIC y el BIC podr\'{i}a depender de varios factores, incluyendo a la funci\'{o}n de p\'{e}rdida, al estudio del dise\~{n}o y la pregunta de investigaci\'{o}n. En modelos de mixtura de variables latentes, la complejidad del modelo verdadero, la separaci\'{o}n de las clases y la proporci\'{o}n de las clases han sido demostradas que afectan la utilidad de estos dos indices de ajuste (Lubke \& Neale, 2006;
	Nylund et al., 2007; Vrieze, 2012). Como mencionamos anteriormente, el BIC actua mejor que el AIC dentro del contexto de modelos de variables latentes.(Jedidi et al., 1997; Li et al., 2009; Nylund et al.,
	2007; Preinerstorfer \& Formann, 2012).                   
	
	
	
	
	

	
	%Simluación
	%---------------------------------------------------- 
\chapter{Estudio de Simulaci\'{o}n}
\label{cap:simulacion}

En este cap\'{i}tulo presentaremos un ejemplo simulado de la aplicaci\'{o}n LTA-DINA. Con el \'{u}ltimo modelo como modelo de medici\'{o}n para evaluar el cambio posterior a una intervenci\'{o}n instruccional educativa. El estudio se basa en un experimento dise\~{n}ado multi et\'{a}pico para evaluar los efectos del tratamiento denominado EAI (instrucci\'{o}n anclada mejorada) en el rendimiento en matem\'{a}ticas en escuelas primarias con alumnos que presentan dificultades de aprendizaje y con aquellos que no la presentan.
Hubieron dos etapas. Una primera aplicaci\'{o}n se realiz\'{o} en el primer semestre acad\'{e}mico seguida de una segunda aplicaci\'{o}n en el mismo a\~{n}o acad\'{e}mico. Los resultados para la segunda evaluaci\'{o}n del FOC (Fraction of the Cost) se presentar\'{a}n en esta simulaci\'{o}n. El impacto de la intervenci\'{o}n EAI en los estudiantes de matem\'{a}tica fu\'{e} examinada al evaluar las diferentes probabilidades de transici\'{o}n seguidas de cada una de las dos aplicaciones instruccionales y evaluando tambi\'{e}n las diferentes probabilidades de transici\'{o}n  entre los diferentes grupos de examinados. 

\noindent
\textbf{Generaci\'{o}n de los datos y dise\~{n}o del estudio}\\
Los estudiantes para este estudio fueron seleccionados de seis salones de matem\'{a}ticas en cierto colegio de una provincia. La muestra consistir\'{a} de 50 varones y 59 mujeres, todos ellos en el s\'{e}timo grado. Nueve de los estudiantes fueron tratados con dificultades de aprendizaje mientras que el resto no presentan estas dificultades.
La prueba FOC fue aplicada cuatro veces, en las semanas 1, 4, 19 y 24 del a\~{n}o acad\'{e}mico. Dos tratamientos instructivos fueron administrados a estos mismos estudiantes durante el a\~{n}o escolar. En el treceavo el instructivo EAI, denominado KK fu\'{e} ense\~{n}ado entre las semanas 1 y 4 y en el onceavo dia el instructivo EAI llamado FOC fu\'{e} aplicado entre las semanas 19 y 24.\\
Entre la aplicaci\'{o}n de estos 2 instructivos, los profesores siguieron su desarrollo regular del s\'{i}labo de matem\'{a}ticas. Cuatro aplicaciones de la prueba FOC fueron realizadas. No hubo en los  estudiantes retroalimentaci\'{o}n en cuanto a las respuestas correctas acerca de su evaluaci\'{o}n con el fin de evitar los efectos posibles de memoria. As\'{i}, los resultados de este estudio demuestran que la estrategia fue exitosa.\\

\noindent
\textbf{Mediciones}\\
La prueba de FOC contiene 20 items de preguntas y respuestas cortas dise\~{n}adas para evaluar los efectos del instructivo FOC. Los \'{i}tems fueron computados dicot\'{o}micamente como correctos e incorrectos. La prueba fue aplicada en cada uno de los cuatro tiempos. Los items en el test fueron construidos para medir la capacidad de los estudiantes para realizar diagramas esquem\'{a}ticos, realizar mediciones, procesar longitudes y determinar costos en la construcci\'{o}n de una rampa para patinadores. La confiabilidad del test fu\'{e} de 0.80. La confiabilidad basada en las cuatro aplicaciones de este estudio fu\'{e} de 0.78, 0.79, 0.79, 0.78 respectivamente.

\noindent
\textbf{Construcci\'{o}n de la Matriz Q}\\
Este paso es muy importante para modelar el diagn\'{o}stico cognitivo y una mala especificaci\'{o}n nos dar\'{a} resultados err\'{o}neos. Para la prueba FOC, fueron medidas 4 habilidades cognitivas: N\'{u}mero y Operaciones, Medici\'{o}n, Resoluci\'{o}n de problemas y Representaci\'{o}n. Un experto en la competencia de educaci\'{o}n matem\'{a}tica identific\'{o} las 4 habilidades e indic\'{o} cuales de estas cuatro habilidades fueron medidas por cada uno de los \'{i}tems en la prueba FOC. Los resultados de este an\'{a}lisis fueron usados para cosntruir la matriz $Q$ con elementos $q_{j k}$ que indica si la habilidad $k$ es necesaria para resolver el \'{i}tem $j$.\\


\begin{table}[H]
	\centering
	\caption{Matriz Q para el test de FOC}
	\vspace{3mm}
	\resizebox{\linewidth}{!}{
		\begin{tabular}{|c|c|c|c|c|}
			\hline
			\textbf{Items} & \textbf{N\'{u}meros y Operaciones} & \textbf{Medici\'{o}n} & \multicolumn{1}{l|}{\textbf{Soluci\'{o}n de Problemas}} & \multicolumn{1}{l|}{\textbf{Representaci\'{o}n}} \\
			\hline
			1 	 & 1	 & 0 	& 0	 & 0 \\ \hline
			2 	 & 1	 & 0 	& 0	 & 0 \\ \hline
			3 	 & 1	 & 1 	& 0	 & 0 \\ \hline
			4 	 & 1	 & 1 	& 0	 & 0 \\ \hline
			5 	 & 1	 & 1 	& 0	 & 0 \\ \hline
			6 	 & 1	 & 1 	& 0	 & 0 \\ \hline
			7 	 & 1	 & 1 	& 0	 & 0 \\ \hline
			8 	 & 1	 & 1 	& 0	 & 0 \\ \hline
			9 	 & 0	 & 1 	& 0	 & 1 \\ \hline
			10	 & 0	 & 1 	& 0	 & 1 \\ \hline
			11	 & 1	 & 1 	& 1	 & 1 \\ \hline
			12	 & 1	 & 1 	& 1	 & 1 \\ \hline
			13	 & 1	 & 1 	& 1	 & 1 \\ \hline
			14	 & 1	 & 1 	& 1	 & 1 \\ \hline
			15	 & 1	 & 1 	& 1	 & 1 \\ \hline
			16	 & 0	 & 0 	& 1	 & 0 \\ \hline
			17	 & 1	 & 1 	& 1	 & 1 \\ \hline
			18	 & 0	 & 1 	& 0	 & 1 \\ \hline
			19	 & 1	 & 0 	& 0	 & 1 \\ \hline
			20	 & 1	 & 0 	& 0	 & 0 \\ \hline
		\end{tabular}
	}
	
\end{table}

\noindent
\textbf{Preguntas de investigaci\'{o}n e hip\'{o}tesis de investigaci\'{o}n basadas en modelos}\\
Describiremos un marco de referencia de hip\'{o}tesis de tal manera que nos permita determinar:(1)si el estado de la habilidad dominada cambiaron a trav\'{e}s de las 4 administraciones de la prueba y si estos cambios subsecuentes a cada intervenci\'{o}n fueron iguales o diferentes. (2) si los diferentes grupos de evaluados tienen diferentes probabilidades de transici\'{o}n. Empezamos asumiendo que las cuatro habilidades son independientes y que el crecimiento en la habilidad es tambi\'{e}n independiente. M\'{a}s adelante asumimos que las probabilidades de transici\'{o}n son diferentes para cada habilidad.
Con respecto a la pregunta 1 hacemos una comparaci\'{o}n b\'{a}sica asumiendo una poblaci\'{o}n homog\'{e}nea con respecto a la matriz de probabilidades de transici\'{o}n.



\begin{table}[H]
	\centering
	\caption{Matriz de transiciones de probabilidad para dos categorias de clases latentes con cuatro puntos en el tiempo}
	\begin{tabular}{lll}
		Test Wave 1 $\rightarrow$ & Test Wave 2 $\rightarrow$ & Test Wave 3 $\rightarrow$ \\
		Test Wave 2 & Test Wave 3 & Test Wave 4 \\ 
		\hline  
		(KK instruction) & Regular instruction & (FOC instruction) \\ 
		\hline 
		$p_{n | n}^{21} p_{m | n }^{21}$ & $p_{n | n}^{32} p_{m | n }^{32}$ & $p_{n | n}^{43} p_{m | n}^{43}$ \\
		$p_{n | m}^{21} p_{m | m }^{21}$ & $p_{n | m}^{32} p_{m | m }^{32}$ & $p_{n | m}^{43} p_{m | m}^{43}$
	\end{tabular}
\end{table}
 
En la tabla anterior $m$ indica logro y $n$ indica no logrado con relaci\'{o}n a la competencia. Como ejemplo consideremos la transici\'{o}n de la primera a la segunda prueba (es decir, el antes y despu\'{e}s en el aplicativo KK). La matriz de probabilidad de transici\'{o}n esta compuesta por cuatro componentes: $p_{n | n}^{21}$ (los que permanecen en el estado de no logrado), $p_{m | m}^{21}$ (los que permanecen en el estado de logrado), $p_{m | n}^{21}$ (los que pasan o transicionan de un estado de logro a un estado no logrado), $p_{n | m}^{21}$ (los que pasan o transicionan de un estado de no logrado a un estado de logrado) 
Las filas de cada matriz de transici\'{o}n son probabilidades condicionales y deben sumar 1, es por eso que para cada matriz de transici\'{o}n, solo se necesitan estimar dos probabilidaes de transici\'{o}n.\\
Tres matrices de probabilidades de transici\'{o}n se construir\'{a}n para observar los efectos del evaluativo KK (Etapa 1 a la Etapa 2), instrucci\'{o}n regular (Etapa 2 a la Etapa 3) y la instrucci\'{o}n del FOC (Etapa 3 a la Etapa 4). Cuatro modelos diferentes de estas tres probabilidades de transici\'{o}n fueron probadas para determinar si los efectos en los tres instructivos diferentes fueron aproximadamente iguales o no.\\

\noindent
\textbf{Modelo A}\\
En este modelo, la matriz de probabilidades de transici\'{o}n para cada etapa se asume diferentes que las otras, de este modo indicamos que los efectos de todos los tres instructivos son diferentes.

$$\left|\begin{array}
{cc}{p_{n | n}^{21}} & {p_{m | n}^{21}} \\ {p_{n | m}^{21}} & {p_{m | m}^{21}}\end{array}\right| \neq\left|\begin{array}{cc}{p_{n | n}^{32}} & {p_{m | n}^{32}} \\ {p_{n | m}^{32}} & {p_{m | m}^{32}}\end{array}\right| \neq\left|\begin{array}{cc}{p_{n | n}^{43}} & {p_{m | n}^{43}} \\ {p_{n | m}^{43}} & {p_{m | m}^{43}}\end{array}\right|$$

\noindent
\textbf{Modelo B}\\
En este modelo, las dos matrices de probabilidades de transici\'{o}n previas al aplicativo FOC se asumen iguales. La transici\'{o}n seguida al FOC, sin embargo se asume que debe ser diferente. Es decir, los efectos del instructivo KK y la aplicaci\'{o}n del instructivo de matem\'{a}tica se asumen iguales, pero ambos difieren del efecto del instructivo FOC.\\

$$\left|\begin{array}
{cc}{p_{n | n}^{21}} & {p_{m | n}^{21}} \\ {p_{n | m}^{21}} & {p_{m | m}^{21}}\end{array}\right|=\left|\begin{array}{cc}{p_{n | n}^{32}} & {p_{m | n}^{32}} \\ {p_{n | m}^{32}} & {p_{m | m}^{32}}\end{array}\right| \neq\left|\begin{array}{cc}{p_{n | n}^{43}} & {p_{m | n}^{43}} \\ {p_{n | m}^{43}} & {p_{m | m}^{43}}\end{array}\right|$$

\noindent
\textbf{Modelo C}\\
En este modelo, la matriz de transici\'{o}n del pre al post test del instructivo KK se asume que debe ser igual a la matriz de transici\'{o}n del pre al post test del instructivo FOC, pero no as\'{i} igual que el pre-post del instructivo regular. Es decir, que los dos tratamientos instructivos EAI que son el KK y el FOC, se asumen que deben afectar a las probabilidades de transici\'{o}n en forma similar, pero ambos difieren de los efectos de la instrucci\'{o}n regular en matem\'{a}ticas.\\

$$\left|\begin{array}
{cc}{p_{n | n}^{21}} & {p_{m | n}^{21}} \\ {p_{n | m}^{21}} & {p_{m | m}^{21}}\end{array}\right|=\left|\begin{array}{cc}{p_{n | n}^{43}} & {p_{m | n}^{43}} \\ {p_{n | m}^{43}} & {p_{m | m}^{43}}\end{array}\right| \neq\left|\begin{array}{cc}{p_{n | n}^{32}} & {p_{m | n}^{32}} \\ {p_{n | m}^{32}} & {p_{m | m}^{32}}\end{array}\right|$$

\noindent
\textbf{Modelo D}\\
En este modelo, todas las matrices de probabilidades de transici\'{o}n se asumen iguales, es decir, todos los evaluativos instruccionales se asumen que deben tener el mismo efecto en los estados de dominio o logro de los estudiantes.

$$\left|\begin{array}
{cc}{p_{n | n}^{21}} & {p_{m | n}^{21}} \\ {p_{n | m}^{21}} & {p_{m | m}^{21}}\end{array}\right|=\left|\begin{array}{cc}{p_{n | n}^{32}} & {p_{m | n}^{32}} \\ {p_{n | m}^{32}} & {p_{m | m}^{32}}\end{array}\right|=\left|\begin{array}{cc}{p_{n | n}^{43}} & {p_{m | n}^{43}} \\ {p_{n | m}^{43}} & {p_{m | m}^{43}}\end{array}\right|$$

Con respecto a la Pregunta 2 que es, si diferentes grupos de examinados tienen diferentes probabilidades de transici\'{o}n, la restricci\'{o}n fue eliminada en el supuesto de homogeneidad de la poblaci\'{o}n, tal que poblaciones diferentes permitieron tener diferentes matrices de probabilidades de transici\'{o}n. Los grupos m\'{a}s importantes ser\'{a}n los estudiantes con discapacidades y aquellos que no presentan estas.
Desafortunadamente, la muestra de estudiantes con dificultades de aprendizaje fue muy peque\~{n}a $(N=9)$ lo que produce poca precisi\'{o}n en las estimaciones de las probabilidades de transici\'{o}n para ese grupo. 

\noindent
\textbf{Estimaci\'{o}n de los par\'{a}metros de los items en el Modelo DINA}\\
En este estudio, el mismo test fue aplicado cuatro veces. Se asume que el efecto memoria no juega un rol importante, toda vez que la prueba consisti\'{o} en una evaluaci\'{o}n del desempe\~{n}o en la cual a los estudiantes se les pregunt\'{o} como realizar las mediciones, el montaje y otras labores para la construcci\'{o}n del rampa de patinaje, y considerando que los estudiantes no recibieron la retroalimentaci\'{o}n acerca de las respuestas correctas despu\'{e}s de cada aplicaci\'{o}n. Adicionalmente, se asume invarianza en los par\'{a}metros del \'{i}tem para el modelo DINAa lo largo de las 4 aplicaciones, a pesar de que las proporciones de dominio en la poblaci\'{o}n se consideran cambiables.
Mas a\'{u}n, los par\'{a}metros de los items se fueron estimados conjuntamente a lo largo de las 4 aplicaciones. Las proporciones de dominio para cada habilidad fueron estimadas libremente a lo largo de las administraciones.\\
El modelo basado en hip\'{o}tesis trata con probabilidades de transici\'{o}n, y como un resultado las estimaciones del par\'{a}metro de los \'{i}tems no afectan demasiado. Las estimaciones de los par\'{a}metros de los \'{i}tems para cada modelo hipot\'{e}tico fueron relativamente muy similares unos a otros.\\
Se presentar\'{a} en el siguiente cuadro las estimaciones de los par\'{a}metros de los \'{i}tems para el Modelo A, las dos primeras columnas son las estimaciones de los prametros al item g y s respectivamente. La tercera columna proporciona estimaciones de la calidad de diagn\'{o}stico de cada \'{i}tem. Esta estimaci\'{o}n se obtiene mediante:

\begin{equation}
	\frac{\left(1-s_{j}\right) / s_{j}}{g_{j} /\left(1-g_{j}\right)}  
	\label{4.1}
\end{equation}           

El cual es llamado el odds ratio entre responder positivamente condicionado a $\eta_{i j}=1$ y de responder positivamente condicionado a $\eta_{i j}=0$. El \'{i}tem con el m\'{a}s alto odds ratio ser\'{a} considerado el m\'{a}s valorable en t\'{e}rminos de la diferenciaci\'{o}n entre las dos clases latentes, donde las dos clases latentes se definen como una primera en la cual los examinados han dominado todas las habilidades requeridas por el \'{i}tem, es decir, $\left.\eta_{i j}=1\right)$ y una segunda en la cual los examinados no han dominado por lo menos una habilidad requerida  por ese \'{i}tem es decir,$\left.\eta_{i j}=0\right)$\\

\noindent
\textbf{M\'{e}todo}\\
Se consider\'{o} un escenario de evaluaci\'{o}n de pre y post test en el cual se aplicaron las mismas pruebas o ex\'{a}menes en dos tiempos($T=2$). Usando los datos generados, se investig\'{o} la recuperaci\'{o}n de par\'{a}metros del modelo LTA-DINA. Particularmente usamos el modelo de diagnostico cognitivo DINA como modelo de medici\'{o}n. Se gener\'{o} respuestas de 1000 estudiantes para 20 items dise\~{n}ados para medir 4 habilidades. La matriz Q se defini\'{o} con la ayuda de expertos y est\'{a} dada en la Tabla 1. Los par\'{a}metros de adivinaci\'{o}n y desliz fueron generados aleatoriamente de una distribuci\'{o}n uniforme entre 0.1 y 0.3, y estos permanecieron constantes a trav\'{e}s del tiempo.

\noindent
Las probabilidades de transici\'{o}n, ser\'{a}n tomadas del articulo de Li.     
El c\'{o}digo R que genera los datos est\'{a} disponible en el Anexo. El modelo fue implementado en el sofware Mplus y el c\'{o}digo tambi\'{e}n se encuentra en el anexo. El criterio de convergencia por defecto usado fue de 0.001 para el cambio absoluto en la logverosimilitud.

\noindent
\textbf{Resultados}\\
 
\begin{table}[H]
	\centering
	\caption{Estimaciones de los par\'{a}metros al item para el modelo LTA-DINA y Odds Ratios.}
	\begin{tabular}{lrrrr}
		\hline
		& \multicolumn{2}{c}{True} & \multicolumn{2}{c}{LTA-DINA}\\
		\hline
		& Guess & Slip & Guess & Slip\\
		\hline
		Item1 & $0.281$ & $0.283$ & $0.20(0.01)$ & $0.23(0.03)$ \\
		Item2 & $0.128$ & $0.287$ & $0.20(0.01)$ & $0.23(0.03)$ \\
		Item3 & $0.300$ & $0.157$ & $0.20(0.01)$ & $0.23(0.03)$ \\
		Item4 & $0.289$ & $0.266$ & $0.20(0.01)$ & $0.23(0.03)$ \\
		Item5 & $0.116$ & $0.228$ & $0.20(0.01)$ & $0.23(0.03)$\\
		Item6 & $0.203$ & $0.204$ & $0.20(0.01)$ & $0.23(0.03)$ \\
		Item7 & $0.178$ & $0.247$ & $0.20(0.01)$ & $0.23(0.03)$ \\
		Item8 & $0.281$ & $0.127$ & $0.20(0.01)$ & $0.23(0.03)$ \\
		Item9 & $0.189$ & $0.231$ & $0.20(0.01)$ & $0.23(0.03)$\\
		Item10 & $0.267$ & $0.241$ & $0.20(0.01)$ & $0.23(0.03)$ \\
		Item11 & $0.247$ & $0.192$ & $0.20(0.01)$ & $0.23(0.03)$ \\
		Item12 & $0.262$ & $0.244$ & $0.20(0.01)$ & $0.23(0.03)$ \\
		Item13 & $0.178$ & $0.287$ & $0.20(0.01)$ & $0.23(0.03)$ \\
		Item14 & $0.237$ & $0.151$ & $0.20(0.01)$ & $0.23(0.03)$ \\
		Item15 & $0.101$ & $0.192$ & $0.20(0.01)$ & $0.23(0.03)$ \\
		Item16 & $0.266$ & $0.288$ & $0.20(0.01)$ & $0.23(0.03)$ \\
		Item17 & $0.101$ & $0.296$ & $0.20(0.01)$ & $0.23(0.03)$ \\
		Item18 & $0.142$ & $0.123$ & $0.20(0.01)$ & $0.23(0.03)$ \\
		Item19 & $0.281$ & $0.195$ & $0.20(0.01)$ & $0.23(0.03)$\\
		Item20 & $0.222$ & $0.212$ & $0.20(0.01)$ & $0.23(0.03)$\\
		\hline
	\end{tabular}
\end{table}
Esto se interpreta como que los items del FOC fueron efecivos al momento de discriminar o distinguir entre los examinados entre las dos clases latentes. Este indice además nos indica que los valores en la matriz Q identifican con certeza las habilidades necesarias para responder bien a los items en la prueba. Los valores grandes de los odds ratios, también fueron debido principalmente a los pequeños parámetros de adivinación. Esto se deduce del hecho de que todos los 20 items de las pruebas FOC fueron de respuestas cortas al item, por consiguiente, la verosimilitud de adivinación fué más baja. 

\begin{table}[H]
	\centering
	\caption{Frecuencia y Proporción de cada habilidad dominada para cada punto en el tiempo.}
	\begin{tabular}{lrr}
		\hline
		Habilidades & \multicolumn{1}{c}{Tiempo 1} & \multicolumn{1}{c}{Tiempo 2}\\
		\hline
		Número y Operaciones 	& $500$ ($0.500$)  & $501$ ($0.501$) \\
		Medición 		& $513$ ($0.513$)  & $513$ ($0.513$) \\
		Solución de problemas		& $496$ ($ 0.496$) & $487$ ($0.487$)\\
		Representación 		& $522$ ($0.522$) & $507$ ($0.507$) \\
		\hline
	\end{tabular}
\end{table}

\noindent
\textbf{Probabilidades de dominio a través de los puntos en el tiempo}\\
La tabla 4.4 presenta frecuencias y proporciones de las personas evaluadas que dominaron cada una de las habilidades cognitivas en cada punto en el tiempo. Las probabilidades de dominio para las 4 habilidades no han aumentado substancialmente inclusive para la habilidades de Solución de problemas y Representación ha disminuído. Es decir al parecer el primer instructivo KK no ha sido adecuado   

\begin{table}[H]
	\centering
	\caption{Frecuencias y Proporciones de los patrones de dominio para las cuatro habilidades en cada punto en el tiempo.}
	\begin{tabular}{lrrrr}
		\hline
		Pattern & \multicolumn{1}{c}{Time Point 1} & \multicolumn{1}{c}{Time Point 2} \\
		\hline
		$(0, 0, 0, 0)$ 	& $188$ ($0.188$) & $189$ ($0.189$) \\
		$(0, 0, 0, 1)$  & $65$ ($0.065$) & $49$ ($0.049$) \\
		$(0, 0, 1, 0)$  & $49$ ($0.049$) & $ 49$ ($0.049$) \\
		$(0, 0, 1, 1)$  & $31$ ($0.031$) & $30$ ($0.030$) \\
		$(0, 1, 0, 0)$ 	& $47$ ($0.047$) & $57$ ($0.030$) \\
		$(0, 1, 0, 1)$  & $41$ ($0.041$) & $43$ ($0.043$) \\
		$(0, 1, 1, 0)$  & $30$ ($0.030$) & $ 30$ ($0.030$) \\
		$(0, 1, 1, 1)$  & $49$ ($0.049$) & $52$ ($0.052$) \\
		$(1, 0, 0, 0)$ 	& $45$ ($0.045$) & $56$ ($0.056$) \\
		$(1, 0, 0, 1)$  & $35$ ($0.035$) & $43$ ($0.043$) \\
		$(1, 0, 1, 0)$  & $32$ ($0.032$) & $ 32$ ($ 0.032$)\\
		$(1, 0, 1, 1)$  & $42$ ($0.042$) & $39$ ($0,039$) \\
		$(1, 1, 0, 0)$  & $32$ ($0.032$) & $35$ ($0.035$)\\
		$(1, 1, 0, 1)$  & $51$ ($0.051$) & $ 41$ ($0.041$)\\
		$(1, 1, 1, 0)$  & $55$ ($0.055$) & $45$ ($0.045$)\\
$(1, 1, 1, 1)$  & $208$ ($0.208$) & $210$ ($0.210$) \\
		\hline
	\end{tabular}
\end{table}
La Tabla 4.5 presenta frecuencias y probabilidades de los patrones de dominio observados en la data del FOC para cada punto en el tiempo. Un patrón que indica el dominio de las cuatro habilidades en un exámen es representado en la Tabla 4.5 como (1,1,1,1). Igualmente un patrón que indica que el examinado dominó ninguna de las cuatro habilidades es representado en la tabla como (0,0,0,0). Las 4 habilidades pueden generar $2^4$ patrones. En datos reales, sin embargo, diferentes dificultades pueden suceder ocasionando en algunos casos que algunos patrones no sean considerados u observados lo cual podria reducir este número de patrones.
Observando los patrones podemos decir que (1,1,1,1) fue el más observado a través de las dos mediciones, pero también el patrón (0,0,0,0) se mantuvo con la misma medición en el tiempo $T=2$.     


\noindent
\textbf{4.1 Estudio de simulación usando los parámetros recuperados de la base de datos FOC}\\
Para confirmar que nuestras estimaciones pueden ser verdaderas o comprobadas, se condujo un estudio de simulación usando los valores de los parámetros estimados de la base de datos FOC. Se simuló datos con las mismas caracteristicas de la data del FOC con la ayuda del software libre R. Las respuestas de 100 estudiantes a 20 items que miden 4 habilidades cognitivas fueron generadas. Se usó la matriz Q de la tabla 4.1, para los parámetros del modelo utilizamos las estimaciones puntuales obtenidas en la data del FOC. Los parámetros de adivinación y desliz fueron aleatoriamente generados a partir de una distribución de probabilidad uniforme desde 0.1 hasta 0.3 y se mantuvieron constantes a través del tiempo.
El modelo LTA-DINA con diferentes probabilidades de transición para cada habilidad y manteniendo constantes los parámetros de adivinación y desliz se muestran en la siguiente tabla:

\begin{table}[H]
	\centering
	\caption{Probabilidades de transición estimadas para el modelo LTA-DINA}
	\begin{tabular}{lrrrrr}
		\hline
		& \multicolumn{1}{c}{$\widehat{p}_{m1}$} & 				\multicolumn{1}{c}{$\widehat{p}_{m|n}$} & 					\multicolumn{1}{c}{$\widehat{p}_{m|m}$} & 
	\multicolumn{1}{c}{$\widehat{p}_{n|m}$} &
	\multicolumn{1}{c}{$\widehat{p}_{n|n}$}\\
		\hline
		Número y Operaciones	 	& ($0.50$) & ($0.256$) &  ($0.256$) &  ($0.255$) & ($0.244$)\\
		Medición			&  ($0.513$) &  ($0.247$) & ($0.247$) & ($0.247$) &  ($ 0.240$)\\
		Solución de problemas		&  ($ 0.496$) &  ($0.231$) &  ($0.231$) &  ($0.240$) & ($0.273$)\\
		Representación		&  ($0.522$) &  ($0.223$) &  ($0.223$) &  ($0.238$) &   ($0.255$)\\
		\hline
	\end{tabular}
\end{table}


\noindent
La tabla 4.6 nos presenta las probabilidades de transición para el modelo LTA-DINA. pm1 indica la probabilidad de dominio de la habilidad en el primer tiempo. pm/n es la probabilidad de transición de un estado de no dominio a un estado de dominio, pm/m es la probabilidad de transición de un estado de dominio a un estado de dominio(es decir, permanencia en el mismo estado), pn/m es la probabilidad de transición de un estado de dominio a un estado de no dominio, pn/n nos indica la probabilidad de transición de un estado de no dominio a un estado de no dominio(permanencia). El modelo además nos dice que para el primer punto en el tiempo, el $50\%$ de los evaluados han logrado el dominio de la habilidad Número y operaciones, el $51.3\%$ de los evaluados han alcanzado a dominar la habilidad de Medición, el $49.6\%$ de los evaluados ha dominado la habilidad Solución de problemas y el $52.2\%$ ha dominado la habilidad de Representación.      

\noindent
pm1 puede ser interpretado como un ratio o razón de aprendizaje, el cual puede ser utilizado como un indicador por ejemplo cuando se evalua una intervención educativa que se realiza en dos periodos en el tiempo.

Además se evaluó el ajuste del modelo. La tabla 4.7 presenta la siguiente información con respecto al ajuste del modelo:  

\begin{table}[H]
	\centering
	\caption{Información del ajuste del modelo LTA-DINA.}
	\begin{tabular}{lrrrr}
		\hline
		Model & \multicolumn{1}{c}{Log-likelihood} & \multicolumn{1}{c}{Number of parameters} & \multicolumn{1}{c}{AIC} & \multicolumn{1}{c}{BIC}\\
		\hline
		LTA-DINA 	& $-16125$  & $52$  & $32355.77$ & $32610.98$  \\
		LTA-DINA 		& $-16125$ & $52$ & $32355.77$  & $32610.98$ \\
		\hline
	\end{tabular}
\end{table}

\noindent
\textbf{4.2 Dominio de las habilidades predichas para el modelo LTA-DINA}\\









   




 



   








   







                 

	
	
	%-------------------------------------------------------------------------------
	\backmatter
	
	%Apéndice	
	\appendix
	
	%Codigo
	\chapter{C\'{o}digo inicial en R}

	\section*{C\'{o}digo en R que genera datos para el ajuste del modelo LTA-DINA}
		\begin{lstlisting}
			# load required packages
			library(MASS)
			library(boot)
			
			# number of respondents
			J <-1000
			
			# number of items
			I <-20
			
			# number of skills
			K <-4
			
			# Q- matrix
			Q <- t(matrix(c(1, 0, 0, 0, 1, 0, 0, 0, 1, 1, 0, 0, 1, 1, 0, 0, 1, 
							  1, 0, 0, 1, 1, 0, 0, 1, 1 ,0, 0, 1, 1, 0, 0, 0, 1, 
							  0, 0, 0, 1, 0, 1, 1, 1, 1, 1, 1, 1 ,1, 1, 1, 1, 1, 
							  1  1, 1, 1, 1, 1, 1, 1, 1, 0, 0, 1, 0, 1, 1 ,1, 1,
							  0, 1, 0, 1, 1, 0, 0, 1, 1, 0, 0, 0) ,K,I))							  
			rownames(Q) <- paste0 (" Item ", 1:I)
			colnames(Q) <- paste0 ("A", 1:K)
			
			# skill profile patterns
			alpha_patt <- as.matrix ( expand.grid (c(0 ,1) ,c(0 ,1) ,c(0 ,1) ,c (0 ,1)))
			colnames(alpha_patt) <- paste0 ("A", 1:4)
			alpha_patt
			
			# slip and guess
			slip <- c(0.192 ,0.260 ,0.119 ,0.291 ,0.143 ,0.182 ,0.237 ,0.209 ,0.134 ,
		         0.241 ,0.238 ,0.206 ,0.279 ,0.164 , 0.266 ,0.256 ,0.118 ,0.291 ,0.210 ,0.264)
			guess <- c(0.201 ,0.242 ,0.263 ,0.122 ,0.230 ,0.186 ,0.119 ,0.117 ,0.174 ,
		          0.205 ,0.274 ,0.123 ,0.265 ,0.278 ,0.293 ,0.233 ,0.133 ,0.165 ,0.150 ,0.283)			
			# generate higher - order latent traits at two time points
			set.seed (1234)
			theta <- mvrnorm(n=J,mu=c(0 ,0.3) , Sigma = matrix (c(1 ,.8 ,.8 ,1) ,2 ,2))
			
			# structural model parameters
			lambda0 <- c(1.51 , -1.42 , -0.66 , 0.5)
			
			# generate true skill mastery profiles and sample responses
			resp <-array(NA , dim =c(J,I ,2))
			A_all <-array(NA , dim=c(J,K ,2))		
			for (t in 1:2){
		 	# find the prob of respondent j having skill k
			 	eta.jk <- matrix(0,J,K)
			 	for (j in 1:J) {
				  	for (k in 1:K){
				    	eta.jk[j, k]<-exp(theta[j,t] + lambda0[k])/(1 + exp(theta [j,t] + lambda0 [k]))}  
				    }
			 		A <- matrix(0,J,K)
			 		for (j in 1:J) {
			  			for (k in 1:K) {
			    			A[j,k]=rbinom(1,1,eta.jk[j,k])
			    		}  
			    	}
			    	
			 		# calculate if respondents have all skills needed for each item
			 		xi_ind <- matrix(0, J, I)
			 		
			 		for (j in 1:J) {
					  	for (i in 1:I) {
					   	 xi_ind[j,i]<-prod(A[j, ]^Q[i, ])
					  	}
				 	}
			 	
			 	# generate prob correct and sample responses
				 prob.correct <- matrix(0, J, I)
			 	y <- matrix(0, J, I)
			 	for (j in 1:J) {
				  	for (i in 1:I) {
					    prob.correct[j,i] <- ((1 - slip [i])^xi_ind [j,i])*( guess [i ]^(1 - xi_ind[j,i]))
				    	y[j, i] <- rbinom(1, 1, prob.correct[j,i])
			    	} 
			    }
			 	
			 	A_all[,,t]<-A
			 	resp [,,t]<-y
			}
			skill_data<-cbind(A_all[,,1],A_all[ , ,2])
			resp_data<-cbind(resp[,,1], resp[ , ,2])
		\end{lstlisting}
	
	\newpage	
	\section*{C\'{o}digo Mplus para el ajuste del modelo LTA-DINA}		
		\begin{verbatim}
			TITLE: !LTA-DINA model for T=2
			DATA:
			FILE IS C:\Users\Usuario\Documents/LTA1_DINA.txt;
			VARIABLE: 
			NAMES ARE
			x1 x2 x3 x4 x5 x6 x7 x8 x9 x10 x11 x12
			x13 x14 x15 x16 x17 x18 x19 x20 y1 y2 
			y3 y4 y5 y6 y7 y8 y9 y10 y11 y12 y13
			y14 y15 y16 y17 y18 y19 y20;
			USEVARIABLES = x1-x20 y1-y20;
			CATEGORICAL = x1-x20 y1-y20;
			CLASSES = c1(2) c2(2) c3(2) c4(2) c5(2) c6(2) c7(2) c8(2);
			ANALYSIS:
			TYPE = MIXTURE;
			PARAMETERIZATION=PROBABILITY;
			STARTS =0;
			ALGORITHM = INTEGRATION;
			PROCESSORS  =  4; 
			MODEL:
			OVERALL
			c5 ON c1;
			c6 ON c2;
			c7 ON c3;
			c8 ON c4;
		\end{verbatim}
		

	
	
	
	
	
	
	
\end{document}