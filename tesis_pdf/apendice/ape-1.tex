\chapter{C\'{o}digo inicial en R}

	\section*{C\'{o}digo en R que genera datos para el ajuste del modelo LTA-DINA}
		\begin{lstlisting}
			# load required packages
			library(MASS)
			library(boot)
			
			# number of respondents
			J <-1000
			
			# number of items
			I <-20
			
			# number of skills
			K <-4
			
			# Q- matrix
			Q <- t(matrix(c(1, 0, 0, 0, 1, 0, 0, 0, 1, 1, 0, 0, 1, 1, 0, 0, 1, 
							  1, 0, 0, 1, 1, 0, 0, 1, 1 ,0, 0, 1, 1, 0, 0, 0, 1, 
							  0, 0, 0, 1, 0, 1, 1, 1, 1, 1, 1, 1 ,1, 1, 1, 1, 1, 
							  1  1, 1, 1, 1, 1, 1, 1, 1, 0, 0, 1, 0, 1, 1 ,1, 1,
							  0, 1, 0, 1, 1, 0, 0, 1, 1, 0, 0, 0) ,K,I))							  
			rownames(Q) <- paste0 (" Item ", 1:I)
			colnames(Q) <- paste0 ("A", 1:K)
			
			# skill profile patterns
			alpha_patt <- as.matrix ( expand.grid (c(0 ,1) ,c(0 ,1) ,c(0 ,1) ,c (0 ,1)))
			colnames(alpha_patt) <- paste0 ("A", 1:4)
			alpha_patt
			
			# slip and guess
			slip <- c(0.192 ,0.260 ,0.119 ,0.291 ,0.143 ,0.182 ,0.237 ,0.209 ,0.134 ,
		         0.241 ,0.238 ,0.206 ,0.279 ,0.164 , 0.266 ,0.256 ,0.118 ,0.291 ,0.210 ,0.264)
			guess <- c(0.201 ,0.242 ,0.263 ,0.122 ,0.230 ,0.186 ,0.119 ,0.117 ,0.174 ,
		          0.205 ,0.274 ,0.123 ,0.265 ,0.278 ,0.293 ,0.233 ,0.133 ,0.165 ,0.150 ,0.283)			
			# generate higher - order latent traits at two time points
			set.seed (1234)
			theta <- mvrnorm(n=J,mu=c(0 ,0.3) , Sigma = matrix (c(1 ,.8 ,.8 ,1) ,2 ,2))
			
			# structural model parameters
			lambda0 <- c(1.51 , -1.42 , -0.66 , 0.5)
			
			# generate true skill mastery profiles and sample responses
			resp <-array(NA , dim =c(J,I ,2))
			A_all <-array(NA , dim=c(J,K ,2))		
			for (t in 1:2){
		 	# find the prob of respondent j having skill k
			 	eta.jk <- matrix(0,J,K)
			 	for (j in 1:J) {
				  	for (k in 1:K){
				    	eta.jk[j, k]<-exp(theta[j,t] + lambda0[k])/(1 + exp(theta [j,t] + lambda0 [k]))}  
				    }
			 		A <- matrix(0,J,K)
			 		for (j in 1:J) {
			  			for (k in 1:K) {
			    			A[j,k]=rbinom(1,1,eta.jk[j,k])
			    		}  
			    	}
			    	
			 		# calculate if respondents have all skills needed for each item
			 		xi_ind <- matrix(0, J, I)
			 		
			 		for (j in 1:J) {
					  	for (i in 1:I) {
					   	 xi_ind[j,i]<-prod(A[j, ]^Q[i, ])
					  	}
				 	}
			 	
			 	# generate prob correct and sample responses
				 prob.correct <- matrix(0, J, I)
			 	y <- matrix(0, J, I)
			 	for (j in 1:J) {
				  	for (i in 1:I) {
					    prob.correct[j,i] <- ((1 - slip [i])^xi_ind [j,i])*( guess [i ]^(1 - xi_ind[j,i]))
				    	y[j, i] <- rbinom(1, 1, prob.correct[j,i])
			    	} 
			    }
			 	
			 	A_all[,,t]<-A
			 	resp [,,t]<-y
			}
			skill_data<-cbind(A_all[,,1],A_all[ , ,2])
			resp_data<-cbind(resp[,,1], resp[ , ,2])
		\end{lstlisting}
	
	\newpage	
	\section*{C\'{o}digo Mplus para el ajuste del modelo LTA-DINA}		
		\begin{verbatim}
			TITLE: !LTA-DINA model for T=2
			DATA:
			FILE IS C:\Users\Usuario\Documents/LTA1_DINA.txt;
			VARIABLE: 
			NAMES ARE
			x1 x2 x3 x4 x5 x6 x7 x8 x9 x10 x11 x12
			x13 x14 x15 x16 x17 x18 x19 x20 y1 y2 
			y3 y4 y5 y6 y7 y8 y9 y10 y11 y12 y13
			y14 y15 y16 y17 y18 y19 y20;
			USEVARIABLES = x1-x20 y1-y20;
			CATEGORICAL = x1-x20 y1-y20;
			CLASSES = c1(2) c2(2) c3(2) c4(2) c5(2) c6(2) c7(2) c8(2);
			ANALYSIS:
			TYPE = MIXTURE;
			PARAMETERIZATION=PROBABILITY;
			STARTS =0;
			ALGORITHM = INTEGRATION;
			PROCESSORS  =  4; 
			MODEL:
			OVERALL
			c5 ON c1;
			c6 ON c2;
			c7 ON c3;
			c8 ON c4;
		\end{verbatim}
		
